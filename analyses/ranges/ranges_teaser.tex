\documentclass[11pt]{article}
\renewcommand{\baselinestretch}{1.8}
\usepackage{textcomp}
\usepackage{fontenc}
\usepackage{graphicx}
\usepackage{caption} % for Fig. captions
\usepackage{gensymb} % for \degree
\usepackage{placeins} % for \images
\usepackage[margin=1in]{geometry} % to set margins
\usepackage{setspace}
\usepackage{lineno}
%\usepackage{cite}
\usepackage{amssymb} % for math symbols
\usepackage{amsmath} % for aligning equations
\usepackage{natbib}

\section*{Background}
There is a general consensus that spring phenology of temperate woody plants is primary cued by forcing, chilling and photoperiod. Yet we (and a number of other studies) find that species differ in their cue use. We now seek to identify the evolutionary and ecological factors that may drive these differences.\\

As a lab we have come up with a few hypotheses--- phylogentic relationships, ecological niches (personified by functional traits), and what is clearly our best idea, that the specifics of the environemental conditions that a species encounters in it native range shape it's cue use. In fact, a number of papers assume range differences among species drive differences phenological cue use, but as far as we can tell this hypothesis has not been rigorously tested.\\

Now, armed with species-level estimates of cue-use from our OSPREE database, grided climate data and species distrubition maps, our team will range for to test this hypothesis.\\ 



\end{document}