\documentclass[11pt,a4paper]{letter}
\usepackage[top=1.00in, bottom=1.0in, left=.75in, right=0.75in]{geometry}
\usepackage{graphicx}
\usepackage{natbib}

\address{1300 Centre Street \\ Boston, MA, 20131}

\begin{document}
\bibliographystyle{/Users/aileneettinger/citations/Bibtex/styles/nature.bst}

\begin{letter}{}
\includegraphics[width=0.5\textwidth]{/Users/aileneettinger/Dropbox/Documents/Work/AA_heading.pdf}
\pagenumbering{gobble}

\opening{Dear Dr. Andrew Sudgen:}
Please consider our paper, entitled ``Chilling outweighs photoperiod and forcing cues for temperate trees in experiments'' for publication as a ``Report" in \emph{Science}. 
\par The timing of budburst in woody plants is critical to fitness, affects wide-ranging ecosystem services from crop productivity to carbon sequestration, and feeds back to global circulation models to inform forecasts of future warming . Temperature (via forcing and chilling) and daylength are cues that determine this timing, known as phenology, in woody plants. 

\par The relative strength of these cues, which will determine how spring phenology shifts with future warming, is a source of controversy. Some research suggests photoperiod cues dominate in particular species or locations and are absent in others (\emph{1-3}), whereas other research argues that winter temperatures (chilling) is critical and reductions in chilling caused by global warming are likely to have strong effects (\emph{4}). Still other research highlights that all three cues may be important for the majority of species (\emph{5}). Reconciling these findings is challenging because studies employ divergent methodologies, across different species and locations, all of which may affect the strength of chilling, forcing, and photoperiod effects.

\par We address this controversy by combining a meta-analytic approach with Bayesian hierarchical modelling to synthesize across nearly four decades of growth chamber experiments and over 200 species around the world. We present a new global database of 66 studies, and our modelling approach allows overall estimates of each cue, as well as species-specific ones. We find most species respond strongly to all three cues, with chilling being the strongest cue (nearly four times greater than forcing). To better understand how the model estimates relate to conditions in the real world, as opposed to experimental ones, we apply our model to well-studied locations in central Europe. Using simple forecasts of warming and historic climate and budburst data, we find that the impact of chilling and daylength cues is highly location-specific---dependent largely on whether chilling increases or decreases with warming. Thus, our results may serve to unify opposing sides of the controversy over phenological cues: while all species may respond to all cues in experimental conditions, the dominant impact of climate change appears to be from increased forcing in current environmental conditions. 

\par Upon acceptance for publication, the database will be freely available at KNB (\emph{6}; currently meta-data are published there); the full database is available to reviewers and editors upon request. This work is a meta-analysis, so data underlying the work have been previously published; however, the compilation and synthesis of these data and the tables, figures, models, and materials presented in this manuscript have not been previously published nor are they under consideration for publication elsewhere.

We recommend the following reviewers: .%Lizzie please add the list that you made at our retreat!


Sincerely,\\

\includegraphics[scale=1]{/Users/aileneettinger/Dropbox/Documents/Work/AileneEttingerSignature.png} \\
Ailene Ettinger
Visiting Researcher, Arnold Arboretum of Harvard University 

\noindent \emph{References mentioned in cover letter}
\begin{footnotesize}
\begin{enumerate}
\item K\"orner, C., \& Basler, D. (2010). Warming, photoperiods, and tree phenology response. \emph{Science}, 329(5989), 278-278.
\item K\"orner, C., \& Basler, D. (2010). Phenology under global warming. \emph{Science}, 327(5972), 1461-1462.
\item Zohner, C. M., et al. "Day length unlikely to constrain climate-driven shifts in leaf-out times of northern woody plants." \emph{Nature Climate Change} 6.12 (2016): 1120.
\item Yu, H., Luedeling, E., \& Xu, J. (2010). Winter and spring warming result in delayed spring phenology on the Tibetan Plateau. \emph{Proceedings of the National Academy of Sciences}, 107(51), 22151-22156.
\item Flynn, D. F. B., \& Wolkovich, E. M. (2018). Temperature and photoperiod drive spring phenology across all species in a temperate forest community. \emph{New Phytologist}, 219(4), 1353-1362.
\item Wolkovich, E., et al. 2019. Observed Spring Phenology Responses in Experimental Environments (OSPREE). Knowledge Network for Biocomplexity. urn:uuid:b2ab2746-b830-436b-a7a9-01b3ef3558e4. 
\end{enumerate}
\end{footnotesize}



\end{letter}
\end{document}



  %  A cover letter which should include
 %       Reference to any pre-submission discussions with editors.
 %       The title of the paper and a statement of its main point.
   %     Any information needed to ensure a fair review process, including related manuscripts submitted to other journals.
     %   Names of colleagues who have reviewed the paper.
        %Specification of where all data underlying the study are available, or will be deposited, and whether there are any restrictions on data availability such as an MTA.

 %       Please also upload a .docx version of your cover letter ? see below.
    %You will have the opportunity to request a specific editor, but this is not required and editor assignment also depends on availability, relative loads and other factors.
    %We require you to list all funding sources. This can be done through a dropdown if your funder is included in FundRef?s controlled vocabulary list.

%Reviewers: Names, affiliations, and e-mail addresses of up to five potential reviewers and up to five excluded reviewers.

