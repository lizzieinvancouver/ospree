

\documentclass[11pt,letter]{article}
\usepackage[top=1.00in, bottom=1.0in, left=1.1in, right=1.1in]{geometry}
\usepackage{graphicx}
\usepackage{natbib}
\usepackage{amsmath}
\usepackage{gensymb} % degree symbol

\def\labelitemi{--}
\parindent=10pt

\begin{document}
%\bibliographystyle{/Users/Lizzie/Documents/EndnoteRelated/Bibtex/styles/besjournals}
\bibliographystyle{..//..//refs/bibstyles/amnat.bst}% i moved a style file into the ospree git repo. feel free to add whatever style you like and update, lizzie! I don't have besjournals

\renewcommand{\refname}{\CHead{}}


{\bf Titles}

Chilling dominates tree budburst in controlled climate experiments, but not in the great outdoors\\
Chilling outweighs photoperiod and forcing cues in temperate trees in experiments, but not in natural systems


\begin{abstract}
Decades of research on woody species highlight how three major cues shape spring phenological events (e.g., budburst and leafout): forcing (warm temperatures, generally occurring in the late winter and early spring), daylength (photoperiod) and chilling (cool temperatures, generally occurring in the fall and late winter). How pervasive these cues are and whether some species are effectively governed by only one or two cues is a critical area of climate change biology research, as it would shape how complex responses to warming will be. Here we use a global meta-analysis of all published growth chamber studies to test for the relative effects of these three major cues across XX species. We find they almost all show these cues, making climate change responses complex. 
\end{abstract}

\section* {Text so far...}

% From Lizzie: I set the \parindent to 10pt in preamble since you use it (I usually just end paragraphs with \\ versus than start them with \par). 
\par For decades, plant phenology has been one of the most reported and consistent biological imprints of climate change \citep{IPCC:2014sm}, with many temperate plants leafing and flowering earlier with rising temperatures (cites). Understanding such shifts is important as phenology shapes a suite of ecosystem services, including pollination and carbon sequestration, and scales up to impact projections of climate change itself. 

\par As research interest in phenology has progressed, critical discrepancies and uncertainties in our understanding have emerged. Though responses to warming are consistent on average, they show high variation across species and sites \citep{Wolkovich:2012n}. Furthermore, long-term observational studies provide increasing evidence that sensitivities of phenology to temperature are weakening in recent decades \citep{yu2010}. In Europe, recent work from many of the most well-studied tree species shows declining responses to temperature, suggesting that the long-term trend towards ever-earlier springs may be stalling \citep{fu2015}. The authors, and others, suggest that responses to other environmental cues underlie these declining temperature sensitivities.

\par Fundamental research in phenology outlines three major cues known to shape spring phenology \citep{chuineJTB}: chilling (cool temperatures, generally occurring in the fall and late winter), forcing (warm temperatures, generally occurring in the late winter and early spring), and daylength (photoperiod). These cues are thought to provide mutiple routes to budburst each spring depending on the environment. For example, a cool winter resulting in chilling above some threshold will require some forcing to trigger budburst, whereas a warmer winter may fail to meet the chilling threshold and thus more forcing, or some combination of forcing and longer daylength, will be required to trigger budburst (cites). Research suggests that all three cues may underlie spring phenology for many temperate woody species (CITES), but there is strong debate, with some research suggesting some cues---such as photoperiod--- may be effectively absent in some species, but dominate in others \citep{zohner2016,koerner2010a}. 

\par Given the declining response to temperature observed in long-term observational studies \citep{fu2015}, a number of papers have tried to tease out evidence of unfufilled chilling or daylength cues in recent years (CITES). This work must overcome the fundamental challenge that all three cues are strongly correlated in nature. During the transition from winter to spring at many temperate latitudes, air temperatures increase (i.e., forcing increases) at the same time that daylength is increasing; likewise, winters with low amounts of chilling are often correlated with warmer springs, and thus higher forcing.  Identifying which of these cues most strongly affects spring phenology is critical for forecasting future phenological changes. For example, if forcing is the dominant cue (as many studies to date have assumed, CITE), then we can expect additional spring advancement as temperatures continue to warm. However, if unfulfilled chilling limits budburst, then we may see delays in spring phenology with additional global warming, which will reduce chilling in many areas \citep{fraga2019}. 

\par In contrast to observational studies, controlled environment experiments can break correlations between chilling, forcing, and photoperiod to reveal which cues underlie budburst phenology. These experiments---most often conducted in growth chambers or similar systems to control temperature and light---have been conducted for decades as a major method to understand the fundamental drivers of spring phenology. To date, controlled environment experiments have identified contrasting effects of the three major budburst cues. Some studies have proposed that photoperiod is likely to constrain species responses to climatic warming \citep{Basler:2012, Caffarra:2011b,Caffarra:2011a}, whereas others report that photoperiod is not a strong cue for most species \citep{zohner2016,Laube:2014a} and suggest chilling may be more important to current and future trends. 

% Note to us: Cat put together a 'what is budburst' file we should include with database. (Full OSPREE: 13,000 rows across 85 studies across 41 years and 227 species)
% We should ONLY report the data that went into the budburst analysis. 
\par Here, we leverage nearly 40 years of controlled environment studies to understand how chilling, forcing, and photoperiod contribute to budburst timing in woody species. Using a meta-analytic approach we reviewed XX papers from controlled environment studies, then extracted data from any papers that met [XX conditions], yielding data from 74 studies across 39 years and 223 species (reference map of studies).  This database includes only studies for which we could identify forcing, photoperiod, and chilling treatments quantitatively. As chilling was only rarely reported, we estimated chilling ourselves when possible using local climate data (see Supplemental Materials). We used a Bayesian hierarchical model to estimate the effects of chilling, forcing, and photoperiod. This model partially pools across species to estimate a robust overall effect, and for robust effects for species with lots of data (\emph{Fagsyl, Betpen}) but pools towards the mean for species with fewer data (See Supplemental Materials- mention species complex).\\

% Watch out for passive voice! Fine sometimes, but we should use it more sparingly.
% Ailene: Can you look into Tilia and Salix showed odd responses? Weird treatments or do papers back this up? We should also check Acer-complex, Fraxinus complex, Cornus alba for forcing ... *after* we fix bbpercdays and get new estimates.
% Also, I changed back to the plan in the outline. It felt like we were leading readers a bit deep into methodological nitty-gritty before getting out the main results: that may be most interesting to phenology readers, but will lose most other folks I fear. 
\par Across studies, all cues---chilling, forcing, and photoperiod---each advance budburst phenology (Fig. \ref {fig:mu}). Using a standardized scale to allow comparisons of the three cues we found that chilling was the strongest cue (-8.3 days/standard unit or -XX days per XX chill portions, Fig. \ref {fig:apc}), followed by forcing (-4.6 days/standard unit or -XX days per degree of warming, Fig. \ref {fig:apc}). Photoperiod was the weakest cue (-3.0 days/standard unit or -XX days per hour); however---given an extensive literature that has suggested photoperiod is a weak or non-existent cue for many species \citep{zohner2016,koerner2010a}--- it was surprisingly large and consistent across species, with only \emph{Fagus sylvatica}, a species well-known for having a large response to photoperiod deviating far from the overall estimate (Figure \ref {fig:mu}). Species also showed fairly consistent responses to chilling (Figure \ref {fig:mu}, though two species delayed budburst with chilling \emph{Tilia codata}, \emph{Salix} complex). Responses to forcing, on the other hand, were the most variable across species (sigma = XX).

% Is the below -- aside from the provenance stuff -- covered in the above (or previously in the intro)? I think so, in which case we can just look out for where to tuck the latitude finding in.
%  \par It has been proposed that photoperiod limitation may also reduce advancement rates of spring budburst for some species \citep{koerner2010a}. We find that photoperiod responses were consistent across species, suggesting that all species rely on this cue for spring phenology \citep{zohner2016, flynn2018,Caffarra:2011a}. Sensitivity to daylength may protect new plant tissues from frost damage by delaying budburst date until a lower risk time of year \citep{koerner2010a}. The magnitude of photoperiod sensitivity varied with provenance latitude: woody material from lower latitudes generally had earlier budburst, given the same chilling, forcing, and daylength conditions (cite supplemental table or figure showing latitude model). %Though we found widespread photoperiod sensitivity across species, the effect was small in comparison to forcing and chilling.

\par As temperature is radically altered by anthropogenic climate change, our finding that different ends of the temperature spectrum---chilling and forcing---have the strongest effects on budburst suggests that understanding these cues will be critical for forecasting. Many previous studies attribute advances in budburst to increased forcing (cites), and forcing sensitivity in our model (-XX days per degree of warming) is consistent with what previous experiments (CITES), and observational studies (CITES) have observed. Our results, however, suggest chilling has a greater effect on budburst than forcing (FIG), which has not been widely suggested previously, as little work has manipulated chilling directly and that work which has usually contrasts it with photoperiod, not forcing (CITES). 
% Need to check: Do any papers compare chilling and forcing cues? Or do they mostly compare photo and chill? 

\par The paucity of studies directly manipulating chilling---which our results suggest has the greatest effect on budburst---suggests a major gap in current research. While many studies (X out of Y here) directly manipulated forcing, far fewer directly manipulated chilling (Z out of Y). Instead many studies (J out of Y) estimate chilling effects through sequential removal of tissue from the field and exposure to `forcing' conditions (cite Weinberger), with the assumption that tissue collected later experienced more chilling. The challenge with this method is that photoperiod and other factors may have also changed during this time. Indeed, we found estimates varied in XX way when derived from direct manipulations of chilling versus the sequential `Weinberger' method. Given the limited manipulation of chilling in studies, we estimated chilling for all studies by combining chilling from the field (i.e., chilling before plants were sampled) and experimental chilling into two widely used metrics of chilling: Utah (citation) and Chill portions (citation). We found the effects of chilling and other cues remain qualitatively the consistent across the two chilling units, though chilling and photoperiod estimates were slightly lower using chill portions compared to Utah (cite supplemental table comparing estimates with both units).  % We could shorten this paragraph further. And then add something about how we need better models of chilling ...

% Note that we don't actually know if tissue is dormant, so we should not say dormant twigs ....
%Most studies do not experimentally apply chilling by manipulating duration or temperature of chilling, nor do most estimate the chilling imposed in their experiment. We therefore calculated the chilling imposed by most studies, as it would otherwise have been impossible to provide estimates with only experimental chilling (reference lizzie's supp heat maps). %this last part may be better for methods?

\par An additional important limitation in most current experiments is the rarity of studies designed to test interactions. While interactions between these cues are widely expected (cites) and, when examined, often found (cites), we were unable to estimate interactions in our meta-analysis because very few studies design experiments to test for interactions between chilling, forcing, and photoperiod (cite table with number of interactions from coding challenge!), and the few that do generally use the Weinberger method, which is not designed to robustly tease out of the effects of multiple cues (cites, Tables, figs).  Our estimated effects average over interations (citeGelman), but identifying them should be an important area of future research. While designing studies to tease out interactions generally requires many additional replicates and careful consideration of the design of treatment combinations, they appear critical to understanding and predicting budburst. For example, the most commonly observed interaction between chilling and forcing---that lower amounts of chilling increase forcing requirements for budburst (cite)---is the hypothesized cause of declining sensitivities in European trees (cites). [Opportunity to sneak in one or two lines on how data limitations meant we could not compare other effects ... then mention latititude x photoperiod?] % Maybe add: In addition to data limitation, disentangling forcing from chilling conditions is a challenge because information on endodormancy requirements is very scarce \citep{chuine2016}.% Though I wonder if this belongs above in discussion of chilling?

% I think the below belongs where we discuss the PEP results. 
\par We expect climate change to continue to have dramatic effects on spring phenology, because the two temperature-derived cues (chilling and forcing)  both strongly affect budburst  \citep{Laube2014a}. However, the relative importance of chilling versus forcing (i.e., the extent to which a chilling threshold will be reached and caused delays in budburst with additional warming) will vary spatially. Forcing is increasing with climate change and is therefore expected to continue advancing budburst. Chilling, however, is expected to increase in some locations and decrease in others with climate change  \citep{fraga2019}, so budburst responses may advance less strongly in places where chilling declines. In some locations, budburst may even delay with substantial amounts of warming (e.g. X degrees, as is forecasted for the end of the 21st century, IPCC, Fig. \ref {fig:fore}, \ref {fig:forefs}) as chilling limitations come into play. 

% I suggest we save a discussion of Photothermoperiodicity for a plant-specific journal in which we have a much larger number of words ...
% Photothermoperiodicity, for example, is an ongoing challenge: chamber studies may seek to replicate patterns in nature, pairing daylength and temperature treatments such that night temperatures are always cooler than day temperatures (e.g., cite studies that do this).  This results in daylength treatments that differ in temperature conditions (and therefore chilling and forcing treatments) as well, however.  


\begin{enumerate}
\item One paragraph: A simple interpretation of our model -- especially its chilling and photo effects -- predicts declining sensitivities in long-term data with climate change. This is because even though forcing increases, chilling is expected to decreases and photoperiods should get shorter -- both predicting delays, and thus an overall muted effect of temperature-only.  (Ref exp conditions forecasting figure.) But how do experimental temperature and photoperiod compare to predicted ones in nature? (Ref experimental conditions forecasting figure)

\item But how do the conditions overlap with natural conditions? (PEP + experimental data figures)
\begin{enumerate}
\item Forcing isn't bad
\item Experimental chilling is generally lower than field chilling
\item Photoperiod differences are very big in experiments
\item Declining sensitivities in PEP data (need to check)
\end{enumerate}

\item Forecasting with these semi-real data, however, do not predict a decline in sensitivity given the moderate amounts of warming already seen, instead they a suggest general advance of budburst until extremely high warming (ref. forecasting figure with PEP-based data)

\begin{enumerate}
\item Chilling often increases with small amounts of warming in some sites
\item Even if warming only happens in the winter, it takes a lot of warming to see a delay due to decreased chilling
\item At higher warming do see a leveling off or delay due to decreased chilling at some sites
\item Depends a lot on local climate... We also find that patterns of advancment with warming vary considerably depending on the current/background climate (e.g. how much advancement will continue with warming depends on how much chilling is currently experienced and whether that will increase or decrease with warming.)
\item (Compare advances in our models to PEP725 data?)
\item Photoperiod effects are minimal, even for \emph{Fagus}
\end{enumerate}

\item So why is PEP725 showing declining sensitivities?

\begin{enumerate}
\item Our results suggest few sites with delays before 3-4 degrees warming (CHECK)... and Germany has warmed X amount
\item Speeding up a biological process given sampling time resolution could lead to declining estimates of sensitivites, even if unchange
\item Say something about what to do about this and how to figure out if this is the issue or it's cues. 
\end{enumerate}

\item Our results suggest most or all studied species are responsive to these three cues
\begin{enumerate}
\item Our results are only for one region, but highlight how critical accurate forecasts of shifts in forcing and chilling will be at local scales
\item To do this, we desperately need to better understand chilling (dormancy release) so that we can predict it in the future (maybe say need better models for chilling across species). 
\item Alongside this, we need more fundamental understanding of interactive cues, which requires larger studies across diverse species. Our results include these complexities but a finer understanding is needed in locations where cues do not change in concert.
\item These complexities are unlikely to alter our fundamental predictions of an increasing advance for many temperate trees in the future, even those with strong chilling or forcing cues (ref Gauzere) [Alt: An improved understanding of interactive cues, however, is unlikely to alter our fundamental predictions of an increasing advance for many temperate trees in the future, even those with strong chilling or forcing cues (ref Gauzere), unless cues are changing very asynchronously.]
\end{enumerate}

\end{enumerate}

\section* {Figures}

\newpage

\begin{figure}[h!]
\centering
\noindent \includegraphics[width=0.75\textwidth]{..//..//analyses/bb_analysis/figures/muplotmodelspcom_expramp_fp_chillports.pdf}
\caption{Budburst model estimates}
\label{fig:mu}
\end{figure}

\begin{figure}[h!]
\centering
\noindent \includegraphics[width=0.75\textwidth]{..//..//analyses/bb_analysis/figures/expcondi_forecastplot.pdf}
\caption{Effects of chilling, forcing, and photoperiod, across the experimental conditions in the OSPREE database. make this part of a 2-panel figure with \ref{fig:mu}. Make a 3D version of this.}
\label{fig:apc}
\end{figure}

\newpage

\begin{figure}[h!]
\centering
\noindent \includegraphics[width=0.75\textwidth]{..//..//analyses/bb_analysis/figures/tempforecast_betpen_minmaxlat_PEPBB_wdl.pdf}
\caption{Implications of global warming on budburst of \emph{Betula pendula} at two locations with differing current climate in Germany, as predicted by our model. Possibly replace with 3D figure? and possibly make a 4-paneled figure with fagsyl}
\label{fig:fore}
\end{figure}
\begin{figure}[h!]
\centering
\noindent \includegraphics[width=0.75\textwidth]{..//..//analyses/bb_analysis/figures/tempforecast_fagsyl_minmaxlat_PEPBB_wdl.pdf}
\caption{Implications of global warming on budburst of \emph{Fagus sylvatica} at two locations with differing current climate in Germany, as predicted by our model. Possibly replace with 3D figure?}
\label{fig:forefs}
\end{figure}

\newpage
\begin{enumerate}
\item $\mu$ plots


\item  $\mu$ forecasting figures: spring x winter warming -- PEP climate range and experimental climate range
\item Species forecasting with PEP data: \emph{Betula, Fagus} ... need to think on which ones to use (x sites x species focus etc). ... Maybe show photoperiod one?
\item PEP data figure with environmental conditions: as in Cat's figure + OSPREE data + maybe foercasting (at 2C or such?)
\end{enumerate}


{\bf Supplemental figures/tables:}
\begin{enumerate}
\item Map of study locations, shading or symbol coding for number of cues (Lizzie)
\item Map of species forecasting to justify sites
\item Tables, yes.
\item Heat maps for the main data, including by actual study design and by calculated chilling (our calculations)
\item Photoperiod x latitude effects figure
\item Equation of our model

\end{enumerate}
\end{enumerate}

\section{Reference list}

A few categories:\\

Papers about contrasting results over what cues matter from growth chamber studies: \cite{Basler:2012,Basler:2014aa,Caffarra:2011qf,Caffarra:2011a,Caffarra:2011b,Heide:2005aa,koerner2010b,Laube:2014a,vitasse2013,zohner2016}. Get Nanninga \emph{et al.} 2017: 'Increased exposure to chilling advances the time to budburst in North American tree species' and maybe Malyshev \emph{et al.} 2018 `Temporal photoperiod sensitivity and forcing requirements for budburst in temperate tree seedlings.'\\

Papers about declining sensitivities (Ailene will update this list): \cite{Rutishauser:2008,fu2015}. Also look for a Wang \emph{et al.} article `Impacts of global warming on phenology of spring leaf unfolding remain stable in the long run.' Vitasse paper on declining variation across elevation gradient. See \cite{yu2010}, but this is not temperate trees. \\

Papers about chilling units paper (Lizzie gets a list): Fu 2012 from OSPREE. \cite{harrington2015}\cite{lued2011,Luedeling:2011qe,Luedeling2013AgFM}\\

\bibliography{..//..//refs/ospreebibplus.bib}
\end{document}