\documentclass[11pt,letter]{article}
\usepackage[top=1.00in, bottom=1.0in, left=1.1in, right=1.1in]{geometry}
\usepackage{graphicx}
\usepackage{natbib}
\usepackage{amsmath}

\def\labelitemi{--}
\parindent=0pt

\begin{document}
\bibliographystyle{/Users/Lizzie/Documents/EndnoteRelated/Bibtex/styles/besjournals}
\renewcommand{\refname}{\CHead{}}


{\bf Titles}

Chilling dominates tree budburst in controlled climate experiments, but not in the great outdoors
Chilling outweighs photoperiod and forcing cues in temperate trees in experiments, but not in natural systems

\section{Outline}

{\bf Figures}

\begin{enumerate}
\item $\mu$ plots
\item Forecasting figures: spring x winter warming -- PEP climate range and experimental climate range
\item Species forecasting: \emph{Betula, Fagus} ... need to think on which ones to use (x sites x species focus etc). ... Maybe show photoperiod one?
\end{enumerate}


{\bf Supplemental figures/tables:}
\begin{enumerate}
\item Map of study locations, shading or symbol coding for number of cues
\item PEP data 
\item Map of species forecasting to justify sites
\item Tables, yes.
\item Heat maps for the main data, including by actual study design and by calculated chilling (our calculations)
\end{enumerate}

{\bf To do:}
\begin{itemize}
\item 3D winter and spring warming
\item Experimental conditions forecasting one
\item Map of study locations, shading or symbol coding for number of cues
\item Heat maps
\item 6 panel draws: 3 temps x 2 sampling regimes (Fake data to examine hypothesis that constant cues with warmer springs could lead to less variability in leafout date, which could in turn lead to smaller estimates of temperature sensitivity.) 
\item Check again this outline versus the budburst files we have (files in earlywriting, budburst.tex)
\end{itemize}


{\bf Outline so far ...}

\begin{enumerate}
\item Why this matters ...climate change.
\begin{enumerate}
\item Climate change alters phenology
\item Increasing evidence of weakening sensitivities
\item We know three cues play a role...
\item but effectively imposible to estimate from long term data... A fundamental challenge of understanding the relative roles of these three cues is that, in the real world, they are often strongly correlated. Chamber experiments often attempt to break this correlation to reveal mechanistic links between environmental conditions and budburst date. 
\item Studies to date appear constrasting (Zohner, Laube etc.)
\item Meta-analysis to the rescue!
\end{enumerate}
\item Budburst phenology is determined by forcing, chilling, and photoperiod. We find that all three cues are important and all three advance budburst.
\begin{enumerate}
\item Chilling is the strongest strongset and most consistent cue (ref Laube and anyone else?)
\item Then forcing, consisent with observational (CITES) and experimental studies (CITES)
\item Then photoperiod.
\end{enumerate}
\item Chilling and forcing
\begin{enumerate}
\item Other interesting stuff!
\item We calculated most of the chilling
\item How you measure chilling matters a bit
\item Weinberger methods is most common for chilling and this is not a super way to measure it.
\end{enumerate}
\item Photoperiod
\begin{enumerate}
\item Photothermoperiodicity
-Increased daylength advances budburst.
-The magnitude of these effects varies by species.
-The magnitude of photoperiod effects varies with latitude, with lower source latitudes generally having earlier budburst.
\end{enumerate}
\item We did not estimate interactions. Why? ..  Add in node to process based models here. 
\begin{enumerate}
\item Very few studies actually do, so there is little to build on
\item The few studies that do interactions often use the weinberger method, which seems a little weird based on our results.
\item They're hard.
\item Our results average over interactive effects. 
\end{enumerate}
\item Species and populations differences 
\begin{enumerate}
\item Latitude matters, latitude matters to photoperiod (provenance)
\item Why species differ and/or why we don't delve into this. 
\end{enumerate}
\item How do these cues translate to climate change (experimental conditions forecasting figure)
\begin{enumerate}
\item Warming winters more important than warming springs (unit for unit)
\item Photoperiod ....
\end{enumerate}
\item But how do the conditions overlap with natural conditions? (PEP + experimental data figures)
\begin{enumerate}
\item Forcing isn't bad
\item Experimental chilling is generally lower than field chilling
\item Photoperiod differences are very big
\end{enumerate}
\item Forecasting with real data suggest general advance of budburst, especially given lower degrees of winter and spring warming
\begin{enumerate}
\item Chilling often increases with small amounts of warming in some sites
\item Even if warming only happens in the winter, it takes a lot of warming to see a delay 
\item At higher warming do see a leveling off or delay due to decreased chilling at some sites
\item Depends a lot on local climate... We also find that patterns of advancment with warming vary considerably depending on the current/background climate (e.g. how much advancement will continue with warming depends on how much chilling is currently experienced and whether that will increase or decrease with warming.)
\item (Compare advances in our models to PEP725 data?)
\item Photoperiod effects are minimal, even for \emph{Fagus}
\end{enumerate}
\item So why is PEP725 showing declining sensitivities?
\begin{enumerate}
\item Our results suggest few sites with delays before 3-4 degrees warming (CHECK)... and Germany has warmed X amount
\item Speeding up a biological process given sampling time resolution could lead to declining estimates of sensitivites, even if unchange
\end{enumerate}
\item Next steps 
\begin{enumerate}
\item We need more studies on interactive cues
\item We desperately need to better understand chilling (dormancy release)
\end{enumerate}
\end{enumerate}

\end{document}


\begin{enumerate}
\end{enumerate}