\documentclass[11.5pt,a4paper]{letter}
\usepackage[top=.75in, bottom=.75in, left=1in, right=1in]{geometry}
\usepackage{graphicx}
\begin{footnotesize}
\address{1300 Centre Street \\ Boston, MA, 20131}
\end{footnotesize}
\begin{document}
\begin{letter}{}
\includegraphics[width=0.3\textwidth]{/Users/aileneettinger/Dropbox/Documents/Work/AA_heading.pdf}
\pagenumbering{gobble}

\opening{Dear Dr. Findlay:}
Please consider our paper, entitled ``Winter temperatures dominate spring phenological responses to warming'' for publication as a ``Letter" in \emph{Nature Climate Change}. This manuscript is a revised version of an earlier submission (NCLIM-19081773). We include a point-by-point response to reviewer comments. 

\par As you may recall, our manuscript utilizes a new global database to address a research topic of critical relevance to a broad reach of \emph{Nature Climate Change} readers:  the timing of spring phenology (e.g., budburst, leafout) in woody plants. Spring phenology impacts plant fitness, shapes plant and animal communities, and affects wide-ranging ecosystem services from crop productivity to carbon sequestration and unites the fields of biometeorology, ecology, cellular and molecular biology. Our work is groundbreaking in its synthesis of four decades of research across 72 experiments
to quantify the relative importance of three environmental cues critical to phenology. We estimate overall chilling, forcing and photoperiod responses for 203 species from around the globe. 

% EMW: I would change the below to be about how we addressed the concerns -- not just list them (i.e., merge with the following paragraph). I can help here.
\par The three reviewers recognized the potential of our work to influence future research. They also highlighted some concerns. Reviewer 1 suggested that additional details and clarification of methods would be beneficial for a fuller evaluation of the study.  Reviewer 2 felt unconvinced that the experimental methods synthesized in our meta-analysis could be reliably applied to natural systems. Reviewer 3 had reservations about the validity of the results given the data and modeling approaches used. 

% EMW: eventually here we want to give some sense in the topic sentence of how much as changed -- changed x% of the text, did Y new analyses, added this many new main text and supp figures. And then also state somewhere that this work has shown our results to be robust (i.e., somewhere you need to say that our big important point still stands). 
\par We have substantially modified the manuscript to address the concerns expressed by reviewers and the issues mentioned by the Editor after the initial submission. Specifically, we have added new text and analyses to the main manuscript, including two new models testing for effects of continent and life stage and as well as applying a recently published modelling approach for estimating temperature sensitivity (sliding windows), as suggested by the reviewer. We have also created a new figure, and modified previous figures in the main text to address reviewer concerns. We have also added substantially to the online `Methods' section, which adheres to the new guidelines of \emph{Nature Climate Change}.

\par Upon acceptance for publication, the database will be freely available at KNB (currently meta-data are there: ADD_KNB_link?); the full database is available to reviewers and editors upon request. This work is a meta-analysis, so data have been previously published; however, the synthesis of these data and the tables, figures, models, and materials presented in this manuscript have not been previously published nor are they under consideration for publication elsewhere.


Sincerely,\\

\includegraphics[scale=.4]{/Users/aileneettinger/Dropbox/Documents/Work/AileneEttingerSignature.png} \\
Ailene Ettinger\\
\begin{footnotesize}
Quantitative Ecologist, The Nature Conservancy- Washington Field Office
Visiting Fellow, Arnold Arboretum of Harvard University 
\end{footnotesize}

\end{letter}
\end{document}
