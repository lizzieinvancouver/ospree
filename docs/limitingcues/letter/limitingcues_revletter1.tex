\documentclass[11pt,letter]{article}
\usepackage[top=1.00in, bottom=1.0in, left=1.1in, right=1.1in]{geometry}
\renewcommand{\baselinestretch}{1.9}
\usepackage{graphicx}
\usepackage{natbib}
\usepackage{amsmath}
\usepackage{amssymb} % for math symbols


\def\labelitemi{--}
\parindent=0pt
\begin{document} 

% START HERE ... 
% open (1) revletter (2)  molecular extras (3) Supp

Questions for co-authors:
1) Do I say 'allowing both statistical estimates of interactions, and insights into the developmental pathways that create them' a little to much? If so, where to cut?
2) How to handle some of R2's questions (photothermal model, fig 2 etc.)?
3) Should we move the chilling figure back in?

TO DO ....
*) fix the formatting in main text and READ IT (send to co-authors? Or start the letter for an hour?)
*) Start the letter.

% For now, just a list of changes ... 

Slights tweaks to abstract: 
\r{absinsights}
\r{absmolec} 

Intro:
\r{ccstudies} 
\r{ccstudiescontrast} 
\r{whatisintxnstart}
\r{whatisintxnstend}
\r{molecpop}

Why obs studies are not enough ...
\r{addreftochillsupp}
\r{smtweakstat1} statistically intxns
\r{smtweakstat2}
\r{whatisintxnstart1}
\r{whatisintxnend1}

\r{newparaonmolecstart}
\r{newparaonmolecend}

New! \subsection{How climate change impacts cues}

\subsection{Interactions alone are unlikely to produce non-linearities with warming}
\r{addmolecstart1}
\r{addmolecend1}

To robustly test for interactions \r{R2comm}*statistically* (line 163)

\emph{Improving controlled environment studies}\\

\r{r2respfx}
\r{r2studydesignstart}
\r{r2studydesignend}

\emph{Incorporating our understanding of physiology into forecasts of phenology}\\
\r{r2moredormstart}
\r{r2moredormend} (some edits in the next paragraph also)


\emph{Improving integration of controlled environment and physiological studies with long-term data}\\
\r{r2acrosscalesstart}
r{r2acrosscalesend}
\emph{Building population- and species-rich predictions} % New title!
\r{r2popstart}
\r{r2popsend}

Supp edits ... 
Relating to lines 140+ ... ``Our search terms and focus on woody species means few of the studies focused on molecular pathways for phenological events, though an extension of this database to include such studies would likely provide important insights in budbreak drivers.''

\end{document}

