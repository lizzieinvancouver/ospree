
% Straight up stealing preamble from Eli Holmes 
%%%%%%%%%%%%%%%%%%%%%%%%%%%%%%%%%%%%%%START PREAMBLE THAT IS THE SAME FOR ALL EXAMPLES
\documentclass{article}
\usepackage{Sweave}
\usepackage{graphicx}
\usepackage{tabularx}
\usepackage{hyperref}
\usepackage{natbib}
\usepackage{pdflscape}
\usepackage{array}
\usepackage{gensymb}
\usepackage{longtable}
\usepackage{xr}
\usepackage{pdflscape}
\usepackage{amsmath}
 \topmargin -1.5cm        
 \oddsidemargin -0.04cm   
 \evensidemargin -0.04cm  % same as oddsidemargin but for left-hand pages
 \textwidth 16.59cm
 \textheight 21.94cm 
 %\pagestyle{empty}       % Uncomment if don't want page numbers
 \parskip 7.2pt           % sets spacing between paragraphs
 %\renewcommand{\baselinestretch}{1.5} 	% Uncomment for 1.5 spacing between lines
\parindent 0pt% sets leading space for paragraphs
\usepackage{setspace}
%\doublespacing
\usepackage{xr}
%\externaldocument{/Users/aileneettinger/Documents/GitHub/ospree/docs/budburst/budburstms}
%\externaldocument{/Users/aileneettinger/Documents/GitHub/ospree/docs/budburst/budburst_supp} 
 
%%%%%%%%%%%%%%%%%%%%%%%%%%%%%%%%%%%%%%END PREAMBLE THAT IS THE SAME FOR ALL EXAMPLES


%Start of the document
\begin{document}

\pagenumbering{gobble}
\setlength\parindent{0pt}

\title{Response to Reviewers}
\emph{Reviewer comments are in italics.} Author responses are in plain text.\\

\emph{{\bf Referee 1, Comments to the Author}}

\par \emph{The review is a well-written summation of how photoperiod may interact with warming temperatures in limiting or altering the spring phenology of plant species. I enjoyed reading the piece, but for a review in New Phytologist, I found it to be rather short and more narrowly focused on spring leaf out and early spring phenology than was obvious from the title and introduction.}

\par \emph{I would suggest that the authors broaden the scope of the work - while the interplay of photoperiod and temperature in controlling spring phenology is less well studied than the role of photoperiod in dictating senescence, the review would have more impact if it incorporated a section on autumn phenology and how similar shifts in time and space may play out there. I'm not suggesting doubling the length of the review, but some explicit discussion of the implications of Figure 1 on autumn phenology and ecology, as only one example, would be welcome and would give the review a much broader audience. Otherwise, though less desirable, the title and abstract should be revised to be be clearer about the focus on spring leaf out.}
\par We thank the reviewer for the suggestion. We have added a paragraph about how the ideas we present may effect autumn phenology, focusing on budset and senescence, as the role of photoperiod on these two autumn phenophases has been well-studied. 

\par \emph{The SI figure could be incorporated into the main text as well. I think it helped drive home the lack of data we have for modeling these responses.}

\par We have moved this figure into the main text.

\par \emph{There is evidence that photoperiod and temperature cues can differentially affect the phenology of different processes in woody species (such as uncoupling of growth or leaf development and carbon uptake [Bauerle et al. 2012 PNAS; Stinziano and Way 2017 PC\&E]). If we incorporate phenological shifts in leaf out into climate models, can we assume that physiological processes are not also impacted separately by photoperiod cues?} 

\par  HELP! Indeed, there are many details and complexiities of both interactive and noninteractive effects of temperature and photoepiod. 
Currently, we do not know what assumptions are reasonable....advocate for exploring this more rigorously. 
???? For spring budburst, photoperiod acts to....(weaker than temp, but would eventually override temp?)

\par \emph{The figures are nicely designed, but not always adequately described in the legend. For example, in Figure 1, it took me quite a while to work out what each arrow was telling me. In the Figure in Box 2, I can't determine why the background figure is even there - two other figures are covering it, and the one on the bottom left is unclear to me as well - I assume it has the same axes and the background figure? A cleaner version of this, separating out the figures, clearly labeling them and providing a full explanation of the graph in the figure is needed.}

\par We thank the reviewer for pointing out a need for more descriptive figure legends and improved versions of Figure 1 and the figure in Box 2.
We have made the following changes:
-Removed the inset from the Box figure


\emph{{\bf Referee 2, Comments to the Author}}
\par \emph{Overall: The review is well written and is presented in a clear and easy to follow structure. I think the inclusion of new analysis should be clearly highlighted.}

\par \emph{Major comments}
\par \emph{Some new analysis and data is included and I think the methods/ datasets should be included in the main manuscript for ease of reading and referencing. Or at least reference the supplementary materials at the points in the text where new analysis is included}
-We have moved sections of the supplement to the main text, and highlighted where new analysis is included in the following places:

\par \emph{Minor comments}
\par \emph{-abstract L5. This is a bit misleading as it is a subset of plant species investigated etc. It may be more accurate to describe photoperiod types of response, LD, SD and day-neutral}

\par HELP!

\par \emph{L120. climate change induced ever earlier springs. Is this correct? What is the data to support it? The statement is made at a global level which may be too broad. Either clarify or provide geographic regions where this is true}

\par \emph{L213. A bit more detail could be included here about the current level of molecular/genetic knowledge}
-Look at Chuine and pull something out to mention here...
\par \emph{L222. highlighted more than shown}


\emph{{\bf Referee 3, Comments to the Author}}
\par \emph{The present manuscript is a review focused on the important topic of the constraint that photoperiod may play in spring phenology dynamics in a warmer world. This is a critical, yet oft overlooked aspect of climate warming impacts on the northern hemisphere. I really appreciated the ideas presented here and think that this review makes an important contribution to this field. I will also note that I jotted down ideas, questions, and comments as I was reading through the paper and repeatedly found that the authors had dealt with these questions or ideas later in the paper! As a consequence, my comments are fairly limited, which is not a function of a surficial review, but rather of a well constructed paper. The writing was clear and logical, making the paper easy to review. Jenn Baltzer}
\par We are grateful for the reviewer's time, positive words, and helpful suggestions.

\par \emph{Line 90: Might this sentence be expanded a bit? Specifically, is there evidence that the potential for northward migrations might be limited by photoperiod constraints? Climate envelope models assume that species will be able shift northward with climate but what if a photoperiod mismatch arises? }
\par We thank the reviewer for this comment. Though we know of no studies documenting photoperiod constraints to range shifts, we did find a recent paper proposing that this might be an increasing concern. We now mention this idea and cite this reference (Lines 90-92 in the revised version):

"For example, poleward shifts in species' ranges cause plants to experience a wider range of daylength throughout the year (Fig. 1), which pay pose challenges to orgainsms undergoing temperature-induced poleward range shifts  (Huffeldt 2020). Elevational shifts, in contrast, cause minimal change to the range of daylength throughout the year.


\par \emph{I really like the comparison presented in figure 1 and described in the paragraph starting on line 97. The example you use in the text is very compelling. }

\par \emph{I appreciated and enjoyed the discussion starting on line 142 regarding differential species sensitive to photoperiod having the potential to lead to community-level shifts (i.e., warming favouring those species that are relatively insensitive to photoperiod). I think that these are really neat ideas. A few thoughts/comments here that could be considered in this section: }
\par \emph{- From your knowledge of this literature, do you think that there are commonalities in the taxa that are less or more sensitive to photoperiod? From a plant productivity perspective, for example, might we expect that species compositional shifts associated with photoperiod sensitivity could also results in systematic shifts in ecosystem productivity, or some other systematic difference in ecosystem function? Given that you have already pulled together a database, this could be quite tractable to explore. For example, it would be neat to combine your OSPREE database that a characterizes species photoperiod sensitivity with TRY database has terrific species-level functional trait data to ask questions about whether certain life history strategies also correspond with photoperiod sensitivity. }
\par \emph{- What about across latitudes? You mention ecotypic divergence in an earlier section in the paper but seems to me that this ecotypic divergence could alter these competitive outcomes as well. For example, aa PhD student of mine conducted a common garden study with black spruce collected across a 2000 km latitudinal gradient (Sniderhan et al. 2018); we grew all trees at a common photoperiod and the high latitude populations would produce a single whorl of growth and then harden off while the more southerly populations just kept on growing until we forced them into a winter. This was not a photoperiod study but this suggests to me that there could be variation across species ranges in their responsiveness to photoperiod that could alter how community dynamics play out under warming.}
\par \emph{- Line 38 reads: ?and the trend of ever-earlier springs with warming may halt?. This assumes that all species are responsive to both temperature and photoperiod cues. If the species differences in responsiveness that you describe there are at play, might this mean that such a halting would not occur? In other words, we might expect certain plant taxa to keep up this trend while others don?t? }
\par perhaps "and the average trend of earlier springs with warming may hald?
\par \emph{You have made beautiful figures that really clearly articulated the problem at hand. A few comments/suggestions: }
\par \emph{1. In Figure 1, you show two shifts for the higher latitude location (early season and peak season) but only one for the lower latitude location. I think it would be useful to show and talk about shifts for both. This would highlight the fact that early season shifts correspond to different relative poleward shifts depending on the latitude. While these differences are apparent at mid-growing season, they would be exaggerated in early season. You may have made this point in the text and I missed it (if so, sorry) but highlighting the fact that warming at higher altitudes results in much more extreme changes in day length is worthwhile given that it is at exactly those latitudes that warming is happening most rapidly. }
\par We thank the reviewer for these suggestions, and have added substantial text to the figure caption to make it stand alone. 

\par \emph{2. In Figure 2, it would be really cool to have a fourth panel: specifically, a trend map for the period between 2009 and 2012 (akin to what is done with NDVI data). Although this figure exemplifies the ideas that are needed for the figure, I think for a review paper that those global trends would be nice to include as well. A short sentence could be added to line 86 to show the magnitude of the trends. }
\par We thank the reviewer for these suggestions, and have added substantial text to the figure caption to make it stand alone. 

\par \emph{3. In my opinion, figure 3 loses impact and interest because it is not stand alone. The reader is required to bounce back and a forth between the paper and the supplement to interpret this figure. I would suggest that the figure caption should be expanded to make the figure stand alone. }
\par We thank the reviewer for these suggestions, and have added substantial text to the figure caption to make it stand alone. 

\par \emph{4. I really like figure 4. }
\par \emph{5. I found the boxes useful additional information. I appreciated the species-specific detail in box 1 that supported the ideas in the main text and box 2 highlights the potential advances in this field that a could be achieved with molecular tools.}
\par We thank the reviewer for these positive comments!
%%%%%%%%%%%%%%%%%%%%%%%%%%%%%%%%%%%%%%%%
\end{document}
%%%%%%%%%%%%%%%%%%%%%%%%%%%%%%%%%%%%%%%%
