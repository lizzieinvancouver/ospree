\documentclass[11pt,a4paper]{article}
\usepackage[top=1.00in, bottom=1.0in, left=1.1in, right=1.1in]{geometry}
\usepackage{graphicx}
\usepackage[sort&compress, super, numbers]{natbib}
\usepackage[export]{adjustbox}

% http://www.nature.com/nclimate/authors/gta/ed-process/index.html

% Researchers should supply a brief paragraph stating the interest to a broad scientific readership, address and contact details, title, a fully referenced summary paragraph, and a list of the references cited in the summary paragraph. Additional material can be included as a separate file if needed.

\begin{document}

\pagenumbering{gobble}

\noindent \includegraphics[width=0.4\textwidth, right]{letterhead/Facultyofforestry.png}
\noindent Dear Dr. Surridge:
\vspace{1.5ex}\\
\noindent Please consider our paper, ``Woody plant phenological responses are strongly associated with key functional traits'' for publication as a research article in \emph{Nature Plants}. 
\vspace{1.5ex}\\ 
\noindent Climate change is impacting species phenologies---timing of life history events---altering ecosystem services and community composition \citep{Cleland2007a,Beard2019,Gu2022}. But predicting these changes is challenging, as it requires an understanding of phenological drivers at a proximate scale---for cues like temperature and daylength---and the ultimate scale at which long-term environmental pressures can produce species differences \citep{Ovaskainen2013,Wolkovich2021}.  Variation in cues may produce gradients in species growth strategies and traits. Early spring species may exhibit traits associated with acquisitive growth that are favourable under abiotically stressful conditions, while late species may exhibit more conservative traits, but greater tolerance to competition. This extension of the existing trait frameworks to include phenology is intuitive, but as a highly variable trait, phenology has often been excluded from trait studies. The relationships between phenology and broader trait syndromes are largely unknown, despite how critical an understanding of these relationships is to accurately forecast future community dynamics.
\vspace{1.5ex}\\
\noindent Here we combined data of experimental leafout phenology and plant traits, with cutting-edge Bayesian approaches, jointly modelling budburst cues in relation to other key traits. Our dataset represents the most comprehensive datasets of trait syndrome available, making it an important first step to identify general trends that scale across populations and species. Further, by using a joint modelling approach, we are the first to identify broader trait relationships to phenological cues based on species-level trait variation, and to account for the high degree of uncertainty that arises when combining datasets of diverse communities.
\vspace{1.5ex}\\
\noindent  Our findings demonstrate how traits and phenologies are inextricably linked to varying strategies for growth, with earlier species also exhibiting acquisitive traits---having short heights, with denser, lower nitrogen leaves---while later-active species were taller with low nitrogen leaves. As such, we found spring leafout phenology to fits within a functional trait framework of acquisitive to conservative growth strategies.
\vspace{1.5ex}\\
\noindent By including phenology in the existing trait framework, we identified the key interactions across traits and cues and can tease apart the underlying mechanisms shaping species phenology across communities. These relationships provide novel insights that can be used to better predict how communities may shift in their growth strategies alongside changing phenology with climate change. Our work also highlights the complexity of interactions shaping plant communities, and represents a more holistic approach that can better forecast future changes in phenologies, community assembly, and productivity of diverse species assemblages.
\vspace{1.5ex}\\
\noindent All authors contributed to this work and approved this version for submission. The manuscript is XXX words with a ZZZ word summary, and X figures. It is not under consideration elsewhere. We hope you find it suitable for publication in \emph{Nature Plants}, and look forward to hearing from you. 
\vspace{1.5ex}\\
\noindent We recommend the following reviewers: Dr. Angela Moles,  Dr. Daniel. C. Laughlin, Dr. Maria Sporbert, Dr. Lee E. Frelich.
\vspace{1.5ex}\\
\noindent Sincerely, \\
\includegraphics[scale=.2]{letterhead/sigDL.png} \\
\noindent Deirdre Loughnan\\
\noindent Forest \& Conservation Sciences\\
\noindent University of British Columbia

\newpage

\vspace{-5ex}
\bibliographystyle{refs/bibstyles/nature.bst}% 

\bibliography{refs/traitors_mar23.bib}
\newpage


\end{document}