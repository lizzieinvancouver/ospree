\documentclass{article}\usepackage[]{graphicx}\usepackage[]{color}

\usepackage{alltt}
\usepackage{float}
\usepackage{graphicx}
\usepackage{tabularx}
\usepackage{siunitx}
\usepackage{amssymb} % for math symbols
\usepackage{amsmath} % for aligning equations
\usepackage{textcomp}
\usepackage{booktabs}
\usepackage{mdframed}
\usepackage{natbib}
\usepackage[colorinlistoftodos]{todonotes} % to make comments on the margin
\usepackage[small]{caption}
\setlength{\captionmargin}{30pt}
\setlength{\abovecaptionskip}{0pt}
\setlength{\belowcaptionskip}{10pt}
\topmargin -1.5cm        
\oddsidemargin -0.04cm   
\evensidemargin -0.04cm
\textwidth 16.59cm
\textheight 21.94cm 
%\pagestyle{empty} %comment if want page numbers
\parskip 7.2pt
\renewcommand{\baselinestretch}{1.5}
\parindent 0pt
%\usepackage{lineno}
%\linenumbers

%% R Script

\title{Insights from functional traits on phenological cue responses in woody temperate plants - Outline}

\begin{document}

\maketitle

\noindent Authors:\\
The Wolkovich Lab in 2021 $^{1,2,3,4}$
\vspace{2ex}\\
\emph{Author affiliations:}\\
$^{1}$Forest \& Conservation Sciences, Faculty of Forestry, University of British Columbia, 2424 Main Mall, Vancouver, BC V6T 1Z4;\\
$^{2}$Arnold Arboretum of Harvard University, 1300 Centre Street, Boston, Massachusetts, USA;\\
$^{3}$Organismic \& Evolutionary Biology, Harvard University, 26 Oxford Street, Cambridge, Massachusetts, USA;\\
$^{4}$Edificio Ciencias, Campus Universitario 28805 Alcalá de Henares, Madrid, Spain\\
 

\vspace{2ex}
$^*$Corresponding author: deirdre.loughnan@alumni.ubc.ca\\
\renewcommand{\thetable}{\arabic{table}}
\renewcommand{\thefigure}{\arabic{figure}}
\renewcommand{\labelitemi}{$-$}
\setkeys{Gin}{width=0.8\textwidth}

Changes in the phenologies of diverse species have become the focus of research into climate change, with widespread trends in species advancing the timing of their life history events. Advances in plant phenological events, such as budburst, leafout, and flowering times in temperate systems are associated with changes in seasonal temperatures, particularly warming winter and spring conditions (Menzel2006...). Despite this strong general trend, considerable variation in phenological responses have been observed across species and geographically (Wolkovich2012). Understanding the relationships between environmental cue and the timing of and variation in phenological events is important for predicting and mitigating species responses to climate change, as well as the effects of these changes on the duration of the growing season, community assembly and ecosystem services (Cleland2007). 

Over the past several decades considerable efforts have been invested in identifying the effects of individual environmental cues on spring phenological events of temperate woody plants. Three cues have been identified as the primary proximate drivers of budburst and leafout -- chilling, forcing, and photoperiod. Winter chilling is defined as the accumulation of chill units under colder temperatures during winter months. High chill requirements are associated with later budburst, however, rising winter temperatures may result in either insufficient chilling in systems where temperatures remain above the required temperature threshold or a faster rate of chilling accumulation with few periods of extremely low temperatures (Guy, 2014). Spring forcing, which is defined as the spring temperatures required by species to initiate budburst, is also changing as temperatures warm and the timing of suitable temperatures for growth occurs earlier in the season (citation). Finally some species have specific requirements for daylength, or photoperiods, to initiate growth. Previous studies support the general trend of advancing budburst (Flynn2018), however, the relative importance of different cues vary across species (Chuine2016?, citations). Some woody plant species for example, will require less forcing to budburst after experiencing a cool winter with more chilling, while also being able to compensate for low chilling in warm winters with greater forcing (Harrington2015). Photoperiod also appears to compensate for low chilling and forcing conditions (Flynn2018,  Caffarra2011, Basler2014,  Zohner2016), however, evidence for the role of photoperiod is largely species specific ( Heide1993, Basler2014, Singh2017, Zohner2016,). Species that are less dependent on photoperiod cues and able to track trends in temperature cues may benefit from greater intra-annual phenotypic plasticity resulting in greater fitness outcomes under increasingly variable climates. 

\end{document}