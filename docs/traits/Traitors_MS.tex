\documentclass{article}\usepackage[]{graphicx}\usepackage[]{color}

\usepackage{alltt}
\usepackage{float}
\usepackage{graphicx}
\usepackage{tabularx}
\usepackage{siunitx}
\usepackage{amssymb} % for math symbols
\usepackage{amsmath} % for aligning equations
\usepackage{textcomp}
\usepackage{booktabs}
\usepackage{mdframed}
\usepackage{natbib}
\usepackage[colorinlistoftodos]{todonotes} % to make comments on the margin
\usepackage[small]{caption}
\setlength{\captionmargin}{30pt}
\setlength{\abovecaptionskip}{0pt}
\setlength{\belowcaptionskip}{10pt}
\topmargin -1.5cm        
\oddsidemargin -0.04cm   
\evensidemargin -0.04cm
\textwidth 16.59cm
\textheight 21.94cm 
%\pagestyle{empty} %comment if want page numbers
\parskip 7.2pt
\renewcommand{\baselinestretch}{1.5}
\parindent 0pt
%\usepackage{lineno}
%\linenumbers

%% R Script

\title{Woody plant phenological responses are strongly associated with key functional traits- Outline}

\begin{document}

\maketitle

\noindent Authors:\\
The Wolkovich Lab in 2021 $^{1,2,3,4}$
\vspace{2ex}\\
\emph{Author affiliations:}\\
$^{1}$Forest \& Conservation Sciences, Faculty of Forestry, University of British Columbia, 2424 Main Mall, Vancouver, BC V6T 1Z4;\\
$^{2}$Arnold Arboretum of Harvard University, 1300 Centre Street, Boston, Massachusetts, USA;\\
$^{3}$Organismic \& Evolutionary Biology, Harvard University, 26 Oxford Street, Cambridge, Massachusetts, USA;\\
$^{4}$Edificio Ciencias, Campus Universitario 28805 Alcalá de Henares, Madrid, Spain\\
 

\vspace{2ex}
$^*$Corresponding author: deirdre.loughnan@alumni.ubc.ca\\
\renewcommand{\thetable}{\arabic{table}}
\renewcommand{\thefigure}{\arabic{figure}}
\renewcommand{\labelitemi}{$-$}
\setkeys{Gin}{width=0.8\textwidth}

Climate change is altering the timing of species phenologies, with changes in temporal niches reshaping ecological communities and interactions between species. In temperate systems, the observed advances in plant phenological events, such as budburst, leafout, and flowering times, are associated with changes in seasonal temperatures, particularly warming winter and spring conditions \citep{Menzel2006,Fitter2002}. But despite this strong general trend, phenological responses vary across species and geographically, and we have yet to fully understand the underlying mechanisms driving observed differences \citep{Chuine2010,Morin2009}. As the effects of climate change become more pronounced, understanding these relationships is of increasing importance if we are to predict and preserve the diversity and services found in temperate forest ecosystems. %The relationships between environmental cues and phenological events define for example, the duration of the growing season, the trajectory of community assembly, species ranges, and ecosystem services \citep{Cleland2007,Lopez2008,Chuine2010}, ultimately shaping and mitigating the impact of climate change on forest communities.

While we have yet to identify all drivers of selection on phenologies, considerable work has shown the importance of three abiotic cues -- chilling, forcing, and photoperiod -- as the primary drivers of budburst and leafout in temperate deciduous species \citep{Basler2014,Chuine2016, Harrington2016,Flynn2018}. For budburst to occur, species must experience extended period of cold temperatures to break dormancy \citep{Cooke2012}, where species with higher chill requirements budburst later in the season. Spring forcing temperatures, or the temperatures needed to cue species to initiate growth after dormancy release, are also changing as temperatures warm and the timing at which suitable temperature thresholds are met occur earlier within the season (citation). Photoperiod cues can also determine a species ability to initiate growth \citep{Basler2014,Zohner2020}. Species with strong photoperiod requirements are, however, expected to be more constrained in their ability to track changes in temperature and may face fitness costs and novel species interactions as a result \citep{Guy2014, others?}. Previous studies support the general trend of advancing budburst in response to each cue, but with considerable variation in the relative importance of different cues across species \citep{Chuine2016,Flynn2018}. Some woody plant species, for example, require less forcing to budburst after experiencing a cool winter with more chilling, while also having the ability to compensate for low chilling with high forcing conditions or longer photoperiods \citep{Laube2014,Harrington2015,Flynn2018,Caffarra2011,Basler2014,Zohner2016}. Evidence for the role of photoperiod is largely species specific  \citep{Heide1993, Basler2014, Singh2017, Zohner2016}, with few studies testing for its importance across species in a community (but see  \cite{Flynn2018}). Species that are less dependent on photoperiod cues and able to track trends in temperatures may benefit from greater intra-annual phenotypic plasticity resulting in greater fitness outcomes under increasingly variable climates (citation?). Despite the insights that identifying these proximate drivers have provided, we still lack a generalizable and mechanistic understanding of why species and populations differ in their cue use that. Further insight on this topic is needed to predict future changes in species sensitivities and community structure.\\

In our efforts to understand variation in spring phenological timing, researchers have tested several potential mechanisms to identify the drivers of species cue responses. Work exploring drivers of intraspecific cue use, for example, has found age or the development stage of woody plants to be important. Younger life stages, including both seedlings and younger understory trees, both budburst earlier than mature individuals in the canopy \citep{Vitasse2013,Seiwa1991}. These trends reflect both differences in the temperature sensitivities across life stages and effects of ontogenic changes as trees mature \citep{Vitasse2013,Seiwa1991}. Interspecific differences in cues, however, have been studied in relation to species' phylogenetic relatedness. Work on this topic has found strong evidence for events like flowering-time and budburst to be consistent within taxonomic families, suggesting conservatism in the genetic and physiological mechanisms that determine species phenologies \citep{Kochmer1986,Davies2013,Gougherty2018}. Studies of woody plant phenologies across species ranges have also highlighted the importance of local adaptations, with the presence of gradients in phenological responses and presumably cue use at northern range limits \citep{Lechowicz1984,Chuine2001,Chuine2010}. In temperate systems for example, greater temperature variation in North America was associated with higher chilling requirements and more conservative phenological responses \citep{Zohner2017}.  Studies testing for trends in cues responses across species latitudinal ranges have also observed stronger responses to photoperiod cues at lower latitudes \citep{Zohner2016}. Exploring these potential drivers of plant phenologies have illustrated the nuanced nature of phenology in shaping diverse communities, but they are still limited in the degree to which they explain the variation we observe across species and ecosystems.\\

Taking a functional trait approach to phenological research could help explain the variation in cue use across species and geographically \citep{Flynn2018,Osada2017}. Early work on functional traits used trait data from diverse global assemblages of deciduous plants to identify associations between traits, common growth strategies, and different niche space \citep{Westoby1998,Wright2004,Chave2009}. The resulting leaf-height-seed scheme and the more extensive leaf economic spectrum found direct associations between several trait values and gradients in species growth rates and competitive abilities \citep{Westoby1998,Wright2004,Diaz2016,Chave2009,Funk2016}. While reproductive phenological traits have been identified as ecologically important for many years \citep{Weiher1999, Laughlin2014}, few studies have explored their role in the larger trait framework. Spring phenological traits, such as budburst and leafout, define the beginning of the growing season and period of photosynthesis, and therefore also have the potential to correlate with established growth strategies. Support for the existence of trade-offs in budburst dates and traits related to growth and resource use have been observed across plant functional groups and habitat types in a handful of studies. For example, several studies have found deciduous woody species with smaller vessel diameters and diffuse or semi-ring-porous xylem structures to leaf out earlier than species with larger vessels, as this anatomy reduces the risk of embolism during freezing events  \citep{Panchen2014, Lechowicz1984}. In testing relationships between budburst and leaf traits of deciduous tree species in Japan, \citep{Osada2017} found positive correlations between budburst date and leaf area, leaf mass, and nitrogen content by both mass and area, while \citep{Sun2006} found deciduous species with high leaf mass per area (a trait that is the inverse of specific leaf area) to budburst earlier in deciduous oak forests in eastern China. %Note their prediction is opposite ours 
Variation in leafout can also relate to species heights, both intraspecifically and across functional groups, with shorter individuals or understory species leafing out earlier than taller individuals or canopy species \citep{Seiwa1998, 1999b}. To date, however, research in this area has focused on individuals at local scales, or few traits for a small number of species, limiting our ability to draw more general and causal inferences. There is also a lack of studies linking traits directly to cue sensitivity rather than phenological date. The likely associations between cue sensitivity, phenological events, and growth strategies may allow for more generalizable trends across species and sites, and better account for species variability in key environmental cue use. \\


%Studies of flowering phenology in herbaceous species also find phenology to relate to commonly measured traits, with early germinating species being taller with greater relative growth rates than later germinating species \citep{Sun2011}. 

% Faith commented that the first two sentences below are repetitive 
To date, there have been numerous studies investigating the relationships between climate and functional traits and a wealth of literature on the separate effects of climate cues as drivers of phenology. However, the selective pressures shaping species traits under variable temperatures are also likely to also act on species responses to phenological cues and define a species temporal niche. Species with a more acquisitive life-strategy  have shorter rates of return on resource investments and the ability to take advantage of the greater abundance of soil nutrients and light early in the growing season. Such species face a lesser cost in triggering phenological events too early, because they can recover more quickly from early season damage (citation?). For example, acquisitive species produce leaves with high leaf nitrogen content and Specific Leaf Area (SLA) and so can take advantage of greater light availability by having higher rates of photosynthesis \citep{Wright2004,Pereira2020}, while also limiting the costs of tissue production \citep{Lambers2004, Westoby2006, Herault2011}. Acquisitive-strategy species also invest less in height and stem density \citep{Laughlin2010}, so will need to trigger the growing season with fewer cue units so they experience less competition for light during budburst. Earlier budbursting species also tend to have lower responses to cues, meaning they they require less spring forcing and winter chilling, and shorter days to trigger the beginning of the season \textbf{(cite Flynn \& Wolkovich, 2018)}. This suite of traits contrasts with more conservative life-strategy species with slower, more competitive growth strategies that benefit from slower rates of return on resource investment and the longer retention of leaf tissue. A greater requirement for cue unit accumulation to trigger phenological events should align with a more conservative life-strategy as such species seek to avoid damage due to premature development.      
 
% Hmmm... is this paragraph still relying heavily on early/late narative?
%

In this study, we test for associations between plant phenological responses to environmental cues and common functional traits. Budburst data for tree species in controlled environmental studies was selected from the Observed Spring Phenology Response in Experimental Environments (OSPREE) database and paired with functional trait data from the TRY and BIEN databases. This data was used to explicitly test for the relative differences in functional traits and the timing of budburst in response to experimental forcing, chilling, and photoperiod cues. Drawing on previous work and the broader trait literature, we predict that species that respond less strongly to chilling, forcing, and photoperiod conditions are more likely to have traits associated with acquisitive growth but low competitiveness, as reflected by high SLA, high LNC, shorter heights, and lower seed mass. In contrast, species that are more responsive to chilling, forcing and photoperiods will have traits more associated with conservative growth and higher competitive abilities, such as low SLA, low LNC, greater heights and heavier seeds. 

%Faith tried to change the two paragraphs above to avoid early/late. Do you think I went to far? I thought our results all relate to the cue slopes rather budburst day so we shoudl avoid mentioning early/late budburst in the hypothesis paragraph. 

\section{Methods}
For our analysis, we combined phenological data from the OSPREE database \citep{OSPREE} with functional trait data from the TRY(cite) and BIEN (cite) trait databases. 

%describe OSPREE breifly
The OSPREE database contains woody, deciduous species phenological data for which experimental data on phenological cues is available, and for which the phylogenetic relationship is well known. First published in 2019, this database has since been updated, and now includes the review of an additional 623 and 270 new publications from each search term respectively. From this subsequent review, we added an additional 12 papers. For additional information on the construction of the OSPREE database and methods of cue estimates, see \citep{OSPREE}. Our analysis used all available budburst data for our 37 focal species, with the data originating from 28 unique studies. 

%Describe TRY and BIEN breifly 
Both TRY and BIEN are large databases compiling plant trait data across many individuals, species, and studies. Initially, we began by selecting height, seed mass, LNC, SLA, stem specific density (SSD) and leaf dry matter content (LDMC) data for all 234 species represented in the OSPREE database.  

%Trait data for ten functional trait was requested from the TRY databases for all 96 species (Table S1 - table of requested traits for each database). Additional trait data was acquired from the BIEN database using the BIEN R package (version X). From the BIEN database we obtained data for 34 species and seven species (Table S1). 

We began by searching for trait data for all 234 species represented in the OSPREE database. Data was also obtained from the BIEN database using the BIEN R package \citep{Maitner2017}. Data was requested or downloaded in December 2018. Our full trait datasets included data on x species .... (S Table x - a tabel only showing trait data for traits we actually used. Not all teh ones we requested.) 

For our analysis we only included trait data from adult individuals with a minimum height of 1.42 m and we removed all data from experiments or growing in non-natural habitats. Traits were also grouped where appropriate, for example, separate entries for specific leaf area (SLA) values with petioles, without petioles, and for which no petiole presence was specified were all categorized as a single trait in our analysis (see Table S1). Duplicated data across the datasets were removed (n= 434905). Finally, we subsetted the data to include only species for which we had a complete dataset for each species and trait. After our selection criteria, out data includes 26 species with at least one measurement for the following six traits: height, seed mass, LNC, SLA, SSD, \& LDMC (n = 60740, n = 404, n = 243, n = 8524, n = 5474, n = 5101 for each trait, respectively). To test for correlations in our six traits and further refine our trait selection, we performed a PCA. The principle component explained 32.2\% of variation while the second explained 23.4\% of the variation (Fig. S1). Given the strong association between the SLA and LDMC leaf traits, and similarly between stem specific density (SSD) and height, we further reduced the number of traits in our analysis to include only height, seed mass, LNC, and SLA. In doing so, we were able to increase the number of species we could include in our analysis, as 37 species had at least one measurement for our final four traits. Given the abundance of height data and overrepresentation of height measurements for six of our focal species, we randomly sampled 3000 height measurements for each of these species to include in our analysis (n = 27318). This reduces the effect of trait values from these frequently measured species from overwhelming the partial pooling effect in our model. In addition we excluded seed mass data from the HE Marx dataset from BIEN, as it consisted of only one value, making it challenging to include the study level effect in our model\\ 

To test the relationships between functional traits and species cue responses, we developed a joint hierarchical bayesian model. Our model is composed of two sub-models that are co-estimated and linked by shared parameters. Because each trait varied in the number of studies in which it is included as well as the number of individuals for which it is measured, we chose to model each trait separately. The first part of the model is a hierarchical intercept only model (Equations \ref{TraitsLine_main}, \ref{TraitsLine_sp} & \ref{TraitsLine_study}) estimating species mean trait value for species $i$ ($trait_{i}$). This ($trait_{i}$) value is a combination of a species mean trait value $\alpha_{sp,i}$ and a hierarchical grouping term on the intercept for study to account for study level differences in the trait data ($\alpha_{study,i}$).  \\

The second part of our models is a hierarchical linear model (Equation \ref{phen_main}) regressing the date of budburst $b$ (pheno_{b}) against a combination of chilling, forcing and photoperiod units. To explicitly compare the effects of chilling, forcing, and photoperiod, we used standardized z-scored values for the predictor variables which accounts for the differences in the scale of predictors across studies \citep{Gelman2006}, as well as the natural units for the cues (including chill units, $^\circ$C, and hours for chilling, forcing, and photoperiod respectively). We test whether there is a link between phenological cue response ($\beta chill_{sp}$,$\beta force_{sp}$,$\beta photo_{sp}$) and mean traits by including the mean trait values from the previous model ($\alpha_{sp}$) in the estimation of the cue slopes. Each cue slope is a combination of a species-specific slope value ($\alpha cue$) independent of trait, and the species trait value ($\alpha_{sp}$) (Equations \ref{alphachilleq}, \ref{alphaForceq} & \ref{alphaPhotoeq})multiplied with an interaction parameter $\beta trait.cue$ (Equations \ref{betaChillEq}, \ref{betaForceEq} & \ref{betaPhotoEq}). A greater $\beta trait.cue$ value means trait is more strongly related to cue slope, and its sign dictates the direction of the interaction. 

Our model was first developed using test data and our priors validated using prior predictive checks. In our models, we used weakly informative priors, with four simultaneous chains of 1,000 warmup iterations and 2,000 sample iterations. The models produced Rhat values close to 1 and neffs greater than 10\% of the number of iterations %triple check this is true
, indicating that the model performed sufficiently well.  We fit our models using the Stan programming language (Stan citation), interfaced with using the rstan package (version, citation).

%should we define every term in our model so we can be spefiic when we discuss them? I (Faith) usually do this, but our model (models?) is so long!

%Our model uses species-level trait values in our first model to predict species sensitivities to forcing, chilling, and photoperiod experimental cues. In addition to including partial pooling across species, the trait portion of the model includes a study level effect, thereby accounting for not only differences across species, but also the effects of methodological differences, and differences across habitats. The first model in our analysis calculates the latent variable that is then incorporated into the second phenology model. Values close to zero reflect small relationships between traits and cues values, while greater values represent high correlations between traits and phenological cues. This model was developed and validated using test data.{}
 
 % How much detail is needed - justify our approach or just include the model?
 %Do we include code for combining the effect of the grand mean with the species level effect?

\begin{equation}
\label{TraitsLine_main}
trait_{i} \sim N( \alpha_{sp,i} + \alpha_{study,i},\sigma_{trait}) 
\end{equation}

\begin{equation}
\label{TraitsLine_sp}
\alpha_{sp} \sim N(\mu, \sigma Sp)
\end{equation}

\begin{equation}
\label{TraitsLine_study}
\alpha_{study} \sim N(0, \sigma Study)
\end{equation} 

\begin{equation}
\label{phen_main}
pheno_{b}  \sim N( \alpha pheno_{sp_b} + \beta force_{sp_b} * Forcing_{b} + \beta photo_{sp_b}  * Photo_{b} + \beta chill_{sp_b} * Chill_{b} , \sigma_{pheno} ) 
\end{equation} 

\begin{equation}
\label{betaChillEq}
\beta chill_{sp} &= \alpha chill_{sp} + \beta trait.chill * \alpha sp_{sp}
\end{equation} 

\begin{equation}
\label{betaForceEq}
\beta force_{sp} &= \alpha force_{sp} + \beta trait.force * \alpha sp_{sp}
\end{equation} 

\begin{equation}
\label{betaPhotoEq}
\beta photo_{sp} &= \alpha photo_{sp} + \beta trait.photo * \alpha sp_{sp}
\end{equation}

\begin{equation}
\label{alphaPheneq}
\alpha pheno & \sim N(\mu_{pheno}, \sigma_{pheno}) 
\end{equation}

\begin{equation}
\label{alphaForceq}
\alpha force& \sim N(\mu_{force}, \sigma_{force}) 
\end{equation}

\begin{equation}
\label{alphachilleq}
\alpha chill & \sim N(\mu_{chill}, \sigma_{chill})
\end{equation}

\begin{equation}
\label{alphaphtotec}
\alpha photo & \sim N(\mu_{photo}, \sigma_{photo}) 
\end{equation}

%\begin{align*}
%\hat{trait_i} &= \mu_{grand_{sp}} + \alpha study_{study_i} \\
%\mu_{grand_{sp}} &= \alpha_{grand} + \alpha sp_{sp_i} \\
%\alpha_{grand}  & \sim N(0, \sigma_{grand})\\
%\alpha sp & \sim N(0, \sigma_{sp}) \\
%\alpha study& \sim N(0, \sigma_{study}) \\
%trait_i & \sim N(\hat{trait}_i, \sigma_{trait}) \\
%\vspace{1ex}\\

%\hat{pheno}_i  &= \alpha pheno_{sp_i} + \beta force_{sp_i} * Forcing_{i} + \beta photo_{sp_i}  * Photo_{i} + \beta chill_{sp_i} * Chill_{i} \\
%\beta force_{sp} &= \alpha force_{sp} + \beta trait.force * \alpha sp_{sp}\\
%\beta chill_{sp} &= \alpha chill_{sp} + \beta trait.chill * \alpha sp_{sp}\\
%\beta photo_{sp} &= \alpha photo_{sp} + \beta trait.photo * \alpha sp_{sp}\\
%\alpha pheno & \sim N(\mu_{pheno}, \sigma_{pheno}) \\
%\alpha force& \sim N(\mu_{force}, \sigma_{force}) \\
%\alpha chill & \sim N(\mu_{chill}, \sigma_{chill}) \\
%\alpha photo & \sim N(\mu_{photo}, \sigma_{photo}) \\
%pheno_{i} & \sim N(\hat{pheno_i}, \sigma_{pheno}) \\
%\end{align*}

%As such, we model each trait individually using the same model specified above, but with the appropriate priors for each trait. Priors were tested using prior predictive checks. All analyses were done in Stan (version) using the rstan package (version) in R (version). 

% I wrote this up in case we want to include it
Finally, we used a phylogenetic greneralized least-squares regression model (PGLS) to test the relationship between day of budburst and individual traits. This analysis allowed us to test for phylogenetic non-independence in the phenology-trait relationship \citep{Freckleton2002}. We obtained a rooted phylogenetic tree by pruning the tree developed by \citep{Smith2018} and performed the PGLS analysis using the mean trait values and mean posterior estimates of the cue responses from our joint model. The PGLS was run using the "Caper" package in R \citep{Orne2013}.


\begin{align*}
y & \sim MVN(\mu, S)
\mu = \mu_{grand_{sp}} + \alpha study_{study_i} \\
\end{align*}

\section{Results}
% I think there are two ways to structure the results - discuss each cue separately or discuss each trait separately - I think each trait is easier and fits our hypotheses better

In general, species cue responses when traits were accounted for were negative, suggesting that forcing, chilling, and photoperiod cues cause the budburst dates to advance. These responses were greatest for forcing and chilling, while the effects of photoperiod cues were more species specific. The effects different traits on cue responses vary across species and for some traits varied with trait values themselves (Figure 1 \& 2).

In our model for SLA, we found all three cue responses to be affected by this trait as well as the trait value itself. Species with low SLA were more sensitive to increased forcing, chilling, and photoperiod cues when the trait effects were included (Fig 2). Species with leaves with high SLA , however, showed no difference in their cue responses when the trait affect was included in the model. Interestingly these species only showed cues responses to forcing and chilling, and were relatively insensitive to changes in photoperiod cues (Fig 2). 

The affect of height on species cue responses was similarly dependent on the magnitude of the trait value. Taller species all showed a stronger trait effect relative to shorter individuals, with the response to both forcing and chilling change from a positive slope to a negative slope when trait values were accounted for. While the effect of photoperiod was less strong, it was also found to only have an effect on taller species. As we observed with SLA, shorter species showed a much smaller trait effect and only advanced in budburst dates in response to chilling and forcing.


Key things to emphasize:
\begin{enumerate}
\item 1. how well the model fits the data
\item 2. Comparison of slopes for full mdl 
\item 3. There is some influence of traits - usually steepens the slopes stronger responses -- taking dark contours and thinking is the slope steeper (more responsive) shallow less sensitive 
\item Can we say anything about the magnitude in the change of bb day?
\end{enumerate}

\begin{itemize}
\item effect of forcing varied with trait values, but generally resulted in earlier budburst with greater forcing
\item across all our models, the effect of photoperiod cues independent of traits were not strong, as illustrated by the flat or negligible slopes - even though data is from experiments with a large range of photoperiod treatments
\item SLA, seedmass, height - the response of cues was independent of hight trait values, traits only mattered for low values
\item High SLA, Ht, seed mass + chill all = advance in bb
\item height mattered regardless of trait value for forcing and chilling
\end{itemize}



Seed mass: different bc log10 - so values can be negative and the effect of a negative trait with the negative slope = positive
Species with high and low seed masses differed in the relative effects of trait on cue use 
Forcing: The importance of traits differed with trait values, low trait species showed no trait effects and inherently advance in their bb with greater forcing. High trait species, have an inherent delaying effect of increased forcing, but this is dampened by the trait effect resulting in a slight delay in bb with increased forcing.

Chilling: Species with small seeds showed a slightly negative response to increased chilling, but in including the effect of seed mass this trend disappeared and species experienced a negligible, positive effect. High trait species differed in their responses, without the effect of seed mass, we found higher chilling to slightly delay bb, but when included our model showed a strong negative trend, with high chilling causing the day of bb to advance. 

Photoperiod: no strong effect of traits, or considerable response to cues. Large seed species showed a slight advance in bb with longer photoperiod, while without the effect of traits there was no response. But small seeded species showed no effect.

LNC: the effect of traits is very small - expect values very close to zero
Forcing: greater forcing leads to an advance in bb regardless of trait values
Chilling: greater chilling leads to an advance in bb regardless of trait values
Photoperiod: no effect of photoperiod 

height:
Forcing: Without the effect of trait, see a slight delay in bb with increasing forcing, but this is reversed for species of all trait values, with all species budbursting earlier when height is accounted for.
Chilling: Without the effect of traits, there is a slight delay in bb with increasing photoperiod, but when height is accounted for, all species advanced in bb with greater chilling regardless of trait values. Day of bb was more variable for Acer pseudoplantanus than Corylus avellana, but overall the model fit the data well, just a lot of variation in these two species
Photoperiod: Without the effect of trait, there is no response to increased photoperiod, but there is an effect when trait value is considered. High trait species show an advance in bb with higher photoperiod. No corresponding trend in low trait species.

How do our findings fit into our predictions?
\begin {itemize}
\item predicted high SLA = respond less to C, P, F - no response to photoperiod, no trait effect on low SLA; when high SLA was accounted for - response to chilling and forcing increased - suggests that species with high SLA will advance in bb with earlier springs and forcing increases, but also may occur in species with high chill requirements?
\item predicted short species to respond less - taller trees were slightly more responsive to photoperiod (like we pred), both have high chilling and forcing responses  (like we predicted), 
\item no effect of LNC
\item predicted lower response to cues for small seeds, no effect of trait for small seeds to forcing or photoperiod, slight delay with chilling, but large seeds - earlier with chilling (as predicted), slightly earlier with longer photoperiod)

\end{itemize}
Drawing on previous work and the broader trait literature, we predict that species that respond less strongly to chilling, forcing, and photoperiod conditions are more likely to have traits associated with acquisitive growth but low competitiveness, as reflected by high SLA, high LNC, shorter heights, and lower seed mass. In contrast, species that are more responsive to chilling, forcing and photoperiods will have traits more associated with conservative growth and higher competitive abilities, such as low SLA, low LNC, greater heights and heavier seeds. 
* use statements that are objectivley ture: less overlap in 50\%, align well, correspondence 
how big the interval is, where they are relative to each other

PGLS suggests there are no strong phylogenetic effects

 We hypothesize that species that budburst under low chilling, low forcing, and short photoperiod conditions are more likely to have traits associated with faster growth, but low competitiveness, as reflected by high SLA, high LNC, shorter heights, and lower seed mass. In contrast, species that budburst later under high chilling or high forcing temperatures, with long photoperiods  may have traits more associated with higher competitive abilities, such as low SLA, low LNC, greater heights and heavier seeds. 

\section{Mean trait values}
Our model generally estimated mean trait values to be close to the simple geometric mean across data, However, there were some cases were the effect of study very was influential, leading to species trait estimates without the effect of study to differ from the simple geometric mean. *show figure of traits posteriors*  
 
\section{Discussion}

How do our results relate to seasonality and frost risk? Is it that early season = risky so more selective pressure for traits to effect cue ues?
Only consider above ground traits

\pagebreak
\bibliographystyle{refs/bibstyles/amnat.bst}% 
\bibliography{refs/traitors.bib}



\end{document}