\documentclass{article}\usepackage[]{graphicx}\usepackage[]{color}

\usepackage{alltt}
\usepackage{float}
\usepackage{graphicx}
\usepackage{tabularx}
\usepackage{siunitx}
\usepackage{amssymb} % for math symbols
\usepackage{amsmath} % for aligning equations
\usepackage{textcomp}
\usepackage{booktabs}
\usepackage{mdframed}
\usepackage{natbib}
\usepackage[colorinlistoftodos]{todonotes} % to make comments on the margin
\usepackage[small]{caption}
\setlength{\captionmargin}{30pt}
\setlength{\abovecaptionskip}{0pt}
\setlength{\belowcaptionskip}{10pt}
\topmargin -1.5cm        
\oddsidemargin -0.04cm   
\evensidemargin -0.04cm
\textwidth 16.59cm
\textheight 21.94cm 
%\pagestyle{empty} %comment if want page numbers
\parskip 7.2pt
\renewcommand{\baselinestretch}{1.5}
\parindent 0pt
%\usepackage{lineno}
%\linenumbers

%% R Script

\title{Insights from functional traits on phenological cue responses in woody temperate plants - Outline}

\begin{document}

\maketitle

\noindent Authors:\\
The Wolkovich Lab in 2021 $^{1,2,3,4}$
\vspace{2ex}\\
\emph{Author affiliations:}\\
$^{1}$Forest \& Conservation Sciences, Faculty of Forestry, University of British Columbia, 2424 Main Mall, Vancouver, BC V6T 1Z4;\\
$^{2}$Arnold Arboretum of Harvard University, 1300 Centre Street, Boston, Massachusetts, USA;\\
$^{3}$Organismic \& Evolutionary Biology, Harvard University, 26 Oxford Street, Cambridge, Massachusetts, USA;\\
$^{4}$Edificio Ciencias, Campus Universitario 28805 Alcalá de Henares, Madrid, Spain\\
 

\vspace{2ex}
$^*$Corresponding author: deirdre.loughnan@alumni.ubc.ca\\
\renewcommand{\thetable}{\arabic{table}}
\renewcommand{\thefigure}{\arabic{figure}}
\renewcommand{\labelitemi}{$-$}
\setkeys{Gin}{width=0.8\textwidth}

Observed changes in the phenologies of diverse species has provided some of the strongest evidence we have for climate change, with widespread trends in species advancing the timing of their life history events. Advances in plant phenological events, such as budburst, leafout, and flowering times in temperate systems are associated with changes in seasonal temperatures, particularly warming winter and spring conditions \citep{MillerRushing2008, Menzel2006, Fitter2002b}. Despite this strong general trend, phenological responses have been observed to vary across species and geographically, and we have yet to fully understand the underlying mechanisms driving these differences \citep{Chuine2010, Morin2009}. As the effects of climate change become more pronounced, understanding these relationships are, of increasing importance. The relationships between environmental cues and phenological events define for example, the duration of the growing season, the trajectory of community assembly, species ranges, and ecosystem services \citep{White1997, Cleland2007, Lopez2008, Chuine2010}, ultimately shaping and mitigating the impact of climate change on forest communities.

Over the past several decades considerable efforts have been made to identify the effects of individual environmental cues on the spring phenological events of temperate woody plants. Three cues -- chilling, forcing, and photoperiod -- are known to be the primary proximate drivers of budburst and leafout \citep{Chuine2016}. For budburst to occur, species must experience extended period of cold temperatures to break dormancy \citep{Cooke2012, Ritchie2003}, with species with higher chill requirements budbursting later in the season. %Rising winter temperatures may result in either insufficient chilling in systems where temperatures remain above the required temperature threshold or a faster rate of chilling accumulation if periods of extremely low temperatures become less frequent \citep{Guy2014}. 
Spring forcing temperatures, or the temperatures needed to cue species to initiate growth after dormancy release, are also changing as temperatures warm and the timing at which suitable temperature thresholds are met occur earlier within the season \citep{citation}. Photoperiod cues can also determine some species ability to initiate growth \citep{BaslerKoner2014, Zohner2020}. Species with strong photoperiod requirements are expected to be more constrained in their ability to track changes in temperature and may face fitness costs and novel species interactions as a result. Previous studies support the general trend of advancing budburst in response to each cue \citep{Flynn2018}, but with considerable variation in the relative importance of different cues across species \citep{Chuine2016?, Flynn2018, citations?}. Some woody plant species, for example, require less forcing to budburst after experiencing a cool winter with more chilling, while also being able to compensate for low chilling with high forcing conditions or longer photoperiods \citep{Laube2014, Harrington2015, Flynn2018,  Caffarra2011, Basler2014,  Zohner2016}. Evidence for the role of photoperiod is largely species specific  \citep{Heide1993, Basler2014, Singh2017, Zohner2016}, with few studies testing for its importance across species in a community (but see  \citep{Flynn2018}. Species that are less dependent on photoperiod cues and able to track trends in temperatures may benefit from greater intra-annual phenotypic plasticity resulting in greater fitness outcomes under increasingly variable climates \citep{citation}. Despite the insights that identifying these proximate drivers have provided, we still lack the generalizable and mechanistic understanding of why species and populations differ in their cue use that is needed to predict future changes in species sensitivities and community structures.

In our efforts to understand variation spring phenologies, a number of hypotheses have been tested to identify the drivers of species cue use. Work exploring drivers of intraspecific cue use, for example, has found age or the development stage of woody plants to be an important factor, with younger life stages, including both seedlings and younger understory trees both budbursting earlier than mature individuals in the canopy \citep{Vitasse2013, Seiwa1999}. These trends reflect both differences in the temperature sensitivities and effects of ontogenic changes as trees mature \citep{Vitasse2013, Seiwa1999}. Interspecific differences in cues in contrast have been studied in relation to species' phylogenetic relatedness. Work on this topic has found strong evidence for events like flowering-time and budburst to be consistent within taxonomic families, suggesting conservatism in the genetic and physiological mechanisms that determine species phenologies \citep{Kochmer1986, Davies2013, Gougherty2018}. Studies of woody plant phenologies across species ranges have also highlighted the importance of local adaptations and the presence of gradients in phenological responses and presumably cue use at northern range limits or in comparing congener species across continents \citep{Lechowicz1984, Chuine2001, Chuine2010}. \citep{Zohner2017}. In temperate systems for example, greater temperature variation in North America was associated with higher chilling requirements and more conservative phenological responses \citep{Zohner2017}.  Across species latitudinal ranges, stronger responses to photoperiod cues have been observed at lower latitudes \citep{Zohner2016}. Exploring these potential drivers of plant phenologies have illustrated the nuanced nature of phenology in shaping diverse communities, but they are still limited in the degree to which they explain the variation we observe across species and ecosystems.

Taking a functional trait approach to phenological research has also been proposed as a promising means and important next step in further explaining the variation in cue use across species and geographically \citep{Flynn2018, Osada2018}. As a functional trait, phenology defines a species' temporal niche and can be used as proxies for tradeoffs in the timing of resource availability and growth, disturbance regimes, or abiotic risk factors. This is inline with several existing axes of variation for other well studied leaf and structural traits. Selection for phenology may therefore be strongly associated with selection for other key functional traits, however, few studies to date have directly tested for these relationships (but see \citep{Osada2018, Sun 2006, Lechowicz1984}. Trait data from diverse global assemblages fo deciduous plants have been used to identify common suites of traits associated with specific growth strategies and niche space \citep{Westoby1998, Wright2004, Chave2009}. Early work on the leaf-height-seed scheme and the leaf economic spectrum, for example, have identified axes in trait space that facilitate fast, more resource acquisitive growth to slower, conservative growth strategies \citep{Westoby1998, Wright2004, Diaz2016, Chave2009, Funk2016}. Species that exhibit faster growth, with shorter rates of return on resource investments, are thought to budburst earlier within a season, when resources are abundant and competition low \citep{citation}. This strategy favours species that can sustain the costs of higher frost risk early in the growing season, or species that possess wood and leaf traits that either mitigate the effects of frost damage, producing leaves at a lower cost if lost \citep{Lechowicz1984, Lenz2013}. Previous studies have found early budbursting species to produce leaves that have high specific leaf areas, with greater investment in leaf area, but more investment in photosynthetic potential and the production of photosynthetic enzymes like Rubisco, as reflected by the high leaf nitrogen content \citep{citation}. Earlier budbursting species experience less competition for light and are expected to be shorter canopy species \citep{citation}. This suite of traits contrast those with slower growth strategies that benefit competitively from slower rates of return on resource investment and the longer retention of leaf tissue \citep{citation}. Phenology has been less frequently incorporated into broader trait studies, however, recent studies have found support for the existence of trade-offs in phenology and traits associated with the aforementioned growth strategies \citep{citations from Kelly's review}. This research has largely consisted of studies conducted within single sites or on few focal species, and are limited in their abilities to draw causal inferences. These likely associations between phenological events and growth strategies may allow for more generalizable trends across species and sites, and better account for species variability in key environmental cue use. 
 
Although there have been numerous studies investigating the relationships between climate and functional traits and a wealth of literature on the effects of climate cues as drivers of phenology, the interrelatedness of traits, phenological responses, and climate drivers has yet to be widely tested. In woody plant species, the associations between budburst, forcing, chilling, and photoperiod cues

 but have rarely been associated with functional traits or the underlying growth strategies these can infer.  Using a diverse global datasets, we explicitly tested for the relative differences in functional traits and budburst responses to forcing, chilling, and photoperiod cues.  

We use global data from the OSPREE database of deciduous woody plant species along with functional trait data from the TRY and BIEN trait databases to identify broad, generalizable trends across species globally. This study focuses on four commonly measured functional traits, SLA, LNC, height, and seed mass, traits that are well represented within the trait databases and known trait schemes, including the leaf economic spectrum \citep{Wright2004} and the classic leaf-height-seed scheme proposed by \cite{Westoby1998}. Drawing on this body of work, we hypothesize that species that budburst under low chilling, low forcing, and short photoperiod conditions may have traits associated with faster growth, but low competitiveness, which may be reflected by high SLA, high LNC, shorter heights, and lower seed mass. In contrast, species that budburst earlier under high chilling or high forcing temperatures, with long photoperiods are predicted to have traits associated with higher competitive abilities, such as low SLA, low LNC, greater heights and heavier seeds. 

\pagebreak
\bibliographystyle{refs/bibstyles/amnat.bst}% 
\bibliography{refs/Proposal.bib}



\end{document}