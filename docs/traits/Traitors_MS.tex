\documentclass{article}\usepackage[]{graphicx}\usepackage[]{color}

\usepackage{alltt}
\usepackage{float}
\usepackage{graphicx}
\usepackage{tabularx}
\usepackage{siunitx}
\usepackage{amssymb} % for math symbols
\usepackage{amsmath} % for aligning equations
\usepackage{textcomp}
\usepackage{booktabs}
\usepackage{mdframed}
\usepackage{natbib}
\usepackage[colorinlistoftodos]{todonotes} % to make comments on the margin
\usepackage[small]{caption}
\setlength{\captionmargin}{30pt}
\setlength{\abovecaptionskip}{0pt}
\setlength{\belowcaptionskip}{10pt}
\topmargin -1.5cm        
\oddsidemargin -0.04cm   
\evensidemargin -0.04cm
\textwidth 16.59cm
\textheight 21.94cm 
%\pagestyle{empty} %comment if want page numbers
\parskip 7.2pt
\renewcommand{\baselinestretch}{1.5}
\parindent 0pt
%\usepackage{lineno}
%\linenumbers

%% R Script

\title{Insights from functional traits on phenological cue responses in woody temperate plants - Outline}

\begin{document}

\maketitle

\noindent Authors:\\
The Wolkovich Lab in 2021 $^{1,2,3,4}$
\vspace{2ex}\\
\emph{Author affiliations:}\\
$^{1}$Forest \& Conservation Sciences, Faculty of Forestry, University of British Columbia, 2424 Main Mall, Vancouver, BC V6T 1Z4;\\
$^{2}$Arnold Arboretum of Harvard University, 1300 Centre Street, Boston, Massachusetts, USA;\\
$^{3}$Organismic \& Evolutionary Biology, Harvard University, 26 Oxford Street, Cambridge, Massachusetts, USA;\\
$^{4}$Edificio Ciencias, Campus Universitario 28805 Alcalá de Henares, Madrid, Spain\\
 

\vspace{2ex}
$^*$Corresponding author: deirdre.loughnan@alumni.ubc.ca\\
\renewcommand{\thetable}{\arabic{table}}
\renewcommand{\thefigure}{\arabic{figure}}
\renewcommand{\labelitemi}{$-$}
\setkeys{Gin}{width=0.8\textwidth}

Changes in the phenologies of diverse species has become some of the strongest evidence we have on climate change, with widespread trends in species advancing the timing of their life history events. Advances in plant phenological events, such as budburst, leafout, and flowering times in temperate systems are associated with changes in seasonal temperatures, particularly warming winter and spring conditions (Menzel2006...). Despite this strong general trend, considerable variation in phenological responses have been observed across species and geographically (Chuine2010, Wolkovich2012). Understanding the relationships between environmental cues and the timing of and variation in phenological events is important for predicting and mitigating species responses to climate change, as well as the effects of these changes on the duration of the growing season, community assembly and ecosystem services (Cleland2007). 

Over the past several decades considerable efforts have been invested in identifying the effects of individual environmental cues on spring phenological events of temperate woody plants. Three cues -- chilling, forcing, and photoperiod -- have are known to be the primary proximate drivers of budburst and leafout. Winter chilling is defined as the accumulation of chill units under colder temperatures during winter months. High chill requirements are associated with later budburst, however, rising winter temperatures may result in either insufficient chilling in systems where temperatures remain above the required temperature threshold or a faster rate of chilling accumulation if periods of extremely low temperatures become less frequent (Guy, 2014). Spring forcing, which is defined as the spring temperatures required by species to initiate budburst, is also changing as temperatures warm and the timing of suitable temperatures for growth occurs earlier in the season (citation). Some species also have specific requirements for daylength, or photoperiod, to initiate growth. Previous studies support the general trend of advancing budburst with each cue (Flynn2018), however, the relative importance of different cues vary across species (Chuine2016?, citations). Some woody plant species, for example, require less forcing to budburst after experiencing a cool winter with more chilling, while also being able to compensate for low chilling in warm winters if high forcing conditions (Harrington2015). Photoperiod also appears to compensate for low chilling and forcing conditions (Flynn2018,  Caffarra2011, Basler2014,  Zohner2016). Evidence for the role of photoperiod is largely species specific (Heide1993, Basler2014, Singh2017, Zohner2016,), with few studies testing for its importance across species in a community (but see Flynn2018). Species that are less dependent on photoperiod cues and able to track trends in temperatures may benefit from greater intra-annual phenotypic plasticity resulting in greater fitness outcomes under increasingly variable climates (citation). Despite the insights that identifying these proximate drivers have provided, we still lack the generalizable and mechanistic understanding of why species and populations differ in their cue use that is needed to predict future changes in species sensitivities and community structures.

In our efforts to understand the observed variation we see in spring phenologies, previous research have aimed to identify key factors in determining species cue use. Intraspecific cue use has been shown to vary with the age or development stage of woody plants. Younger life stages, including both seedlings and younger understory trees both budburst earlier than mature individuals in the canopy, reflecting both differences in their temperature sensitivities and more ontogenic changes as trees mature (Vitasse2013, Seiwa1999). Differences in phenologies across species have also been assessed in relation to a species phylogenetic relatedness. These studies suggest that within taxonomic families, events like flowering-time and budburst are consistent and reflect phylogenetic conservatism in the genetic and physiological mechanisms that determine species phenologies (Kochmer1986, Davies2013). Other work has focused on the relationships between phenology and geographic gradients, including latitude and range limits, based on the assumption that woody plant phenologies reflect local adaptations to climate cues within a species native range (Lechowicz1984, Chuine2001, Chuine2010). Zohner2017 observed associations between greater temperature variation in the spring and higher chilling requirements of species in North America compared to species native to Europe and Asia. More variable spring conditions in North America are hypothesized to have selected for more conservative phenological responses as a means of mitigating disturbance events such as late season frosts (citations). Across species latitudinal ranges, stronger responses to photoperiod cues have been observed at lower latitudes. Exploring these potential drivers of plant phenologies have added to our understanding of nuanced nature of phenology in shaping diverse communities. They still, however, are limited in the degree to which they explain the variation we observe across species and ecosystems.

Previous work has demonstrated the importance of both the evolutionary history of a species and environmental cues in shaping species phenologies and therefore the temporal niche of a species. However, few studies have directly explored the potential for the drivers in this niche partitioning to explain variation in cue use. Trade-offs in the timing of resource availability and growth, disturbance regimes, or abiotic risk factors likely also shape species phenologies. Plant functional traits have been identified as good proxies for these tradeoffs and have the potential to offer new insights into the relationships between climate, environmentally determined phenological events like budburst, and gradients in functional traits.









\end{document}