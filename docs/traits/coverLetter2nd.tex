\documentclass[11pt,a4paper]{article}
\usepackage[top=1.00in, bottom=1.0in, left=1.1in, right=1.1in]{geometry}
\usepackage{graphicx}
\usepackage[sort&compress, super, numbers]{natbib}
\usepackage[export]{adjustbox}

% 50 word answers
\begin{document}

\pagenumbering{gobble}
%dlOct10: is "shape" to strong a word? It implies causation in my mind. Should we say "how budbrust timing relates to woody plant strategies and traits"?
\noindent \includegraphics[width=0.4\textwidth, right]{letterhead/ubc-logo-2018-fullsig-blue-cmyk.png}

\noindent Dear Dr. Bardgett,
\vspace{1.5ex}\\
\noindent Please consider our revised manuscript, ``Budburst timing within a functional trait framework'' for publication as a research article in \emph{Journal of Ecology}. 
\vspace{1.5ex}\\ 
\noindent The timing of plant life history events---phenologies---has shifted with climate change, altering the availability of resources and severity of competition and thereby the overall community dynamics and ecosystem services. But to date, few studies have explored whether changes in phenological traits relate to other traits that are known to relate to resource use and competition. Our work is the first to use data from global trait and phenology databases to test how woody plant budburst timing relates to other major functional traits and well established frameworks of species growth strategies. We found species that budburst earlier exhibit traits associated with an acquisitive growth strategy---shorter maximum heights, and high nitrogen leaves---while later budbursting species are taller with a low nitrogen content, indicative of a more conservative strategy that is favourable later in the growing season. In showing how budburst relates to traits from established functional trait frameworks, our results provide the foundation needed to predict how continued shifts in phenology may alter community dynamics under future conditions. 
\vspace{1.5ex}\\ 
Two reviewers highlighted the strengths of our analysis, finding the approach we took ``very exciting" and the manuscript ``well written." Their comments, questions and suggested corrections have greatly improved the manuscript. In response to the reviewer comments, we have refined our language and provided greater support and clarification of our predictions. We have revised the introduction extensively. It now has a more in-depth overview of the literature that supports our predictions and provides more context for our results. We have also created two new figures to better visualize the global scope of the data and to provide more detail on our methods for data collection and cleaning. We have made considerable revisions to our results and discussion to improve their clarity and interpretability, including the expansion of our tables for each model in the supplementary material and more detailed explanations in both the text and captions. Finally we added additional context and discussion regarding our analysis of species versus study-level variation to better highlight the importance of partitioning these sources of variation to make accurate predictions.
\vspace{1.5ex}\\
We believe the new manuscript has been greatly improved, as discussed in our point-by-point responses. The manuscript is 3569 words with a 309 word summary and three figures. This article is not under consideration for publication elsewhere. We hope you find it suitable for publication in \emph{Journal of Ecology}, and look forward to hearing from you. 
\vspace{1.5ex}\\
\noindent Sincerely, \\
\includegraphics[scale=.4]{letterhead/shot.png} \\ 
\noindent Deirdre Loughnan\\
\noindent Sentinels of Change Postdoctoral Fellow\\ %emw19Dec: Nice!
\noindent Hakai Institute $|$ Department of Zoology\\
\noindent University of British Columbia
\newpage

\end{document}

% do I have too much going on here? Chunky sentences? Can they be simplified?