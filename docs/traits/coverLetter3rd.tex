\documentclass[11pt,a4paper]{article}
\usepackage[top=1.00in, bottom=1.0in, left=1.1in, right=1.1in]{geometry}
\usepackage{graphicx}
\usepackage[sort&compress, super, numbers]{natbib}
\usepackage[export]{adjustbox}

% 50 word answers
\begin{document}

\pagenumbering{gobble}
%dlOct10: is "shape" to strong a word? It implies causation in my mind. Should we say "how budbrust timing relates to woody plant strategies and traits"?
\noindent \includegraphics[width=0.4\textwidth, right]{letterhead/ubc-logo-2018-fullsig-blue-cmyk.png}

\noindent Dear Dr. Bardgett,
\vspace{1.5ex}\\
\noindent Please consider our revised manuscript, ``Budburst timing within a functional trait framework'' for publication as a research article in \emph{Journal of Ecology}. 
\vspace{1.5ex}\\ 
\noindent The timing of plant life history events (phenologies) has shifted with climate change, altering the availability of resources and severity of competition and thereby the overall community dynamics and ecosystem services. But to date, few studies have explored whether changes in phenological traits relate to other traits that are known to relate to resource use and competition. Our work is the first to use data from global trait and phenology databases to test how woody plant budburst timing relates to other major functional traits and well established frameworks of species growth strategies. We found species that budburst earlier exhibit traits associated with an acquisitive growth strategy, including shorter maximum heights, and high nitrogen leaves, while later budbursting species are taller with a low nitrogen content, indicative of a more conservative strategy that is favourable later in the growing season. In showing how budburst relates to traits from established functional trait frameworks, our results provide the foundation needed to predict how continued shifts in phenology may alter community dynamics under future conditions. 
\vspace{1.5ex}\\ 
We are happy to hear that the previous revisions addressed the concerns of our previous reviewer and that a third reviewer finds our work to be "an interesting topic with current relevance" that "advances the current understanding of budburst response to climate change."  In addressing the reviewer comments, we have made further improvements to the manuscript. We have provided greater detail to the introduction and discussion elaborating on how we define small versus large responses. We also revised our main figure to clearly highlight our main results and include annotations that help guide readers in their interpretation. We extensively revised the structure of our writing throughout the manuscript to improve the flow of ideas and improve reader comprehension. Finally we added an additional subsection to the discussion and elaborated on the considerable applications of our methods and general approach to other areas of ecology and conservation.
\vspace{1.5ex}\\
We believe the new manuscript has been greatly improved, as discussed in our point-by-point responses. The manuscript is 4536 words with a 301 word summary and three figures. This article is not under consideration for publication elsewhere. We hope you find it suitable for publication in \emph{Journal of Ecology}, and look forward to hearing from you. 
\vspace{1.5ex}\\
\noindent Sincerely, \\
\includegraphics[scale=.4]{letterhead/shot.png} \\ 
\noindent Deirdre Loughnan\\
\noindent Sentinels of Change Postdoctoral Fellow\\ %emw19Dec: Nice!
\noindent Hakai Institute $|$ Department of Zoology\\
\noindent University of British Columbia
\newpage

\end{document}

% do I have too much going on here? Chunky sentences? Can they be simplified?