\documentclass[11pt,a4paper]{article}
\usepackage[top=1.00in, bottom=1.0in, left=1.1in, right=1.1in]{geometry}
\usepackage{graphicx}
\usepackage[sort&compress, super, numbers]{natbib}
\usepackage[export]{adjustbox}

% 50 word answers
\begin{document}

\pagenumbering{gobble}
%dlOct10: is "shape" to strong a word? It implies causation in my mind. Should we say "how budbrust timing relates to woody plant strategies and traits"?
\noindent \includegraphics[width=0.4\textwidth, right]{letterhead/ubc-logo-2018-fullsig-blue-cmyk.png}

\noindent Dear Dr. Bardgett,
\vspace{1.5ex}\\
\noindent Please consider our revised manuscript, ``Budburst timing within a functional trait framework'' for publication as a research article in \emph{Journal of Ecology}. 
\vspace{1.5ex}\\  % it's timing -- it's singular here; also this sentence is dense; I reduced 'altering the availability of resources and severity of competition and thereby the overall community dynamics and ecosystem services' but you should work to flag for yourself these types of sentences (a sentence with 'and ... and thereby ...' is rarely easy on the reader). I think this is not a big deal since we submitted before. 
\noindent The timing of plant life history events (phenology) has shifted with climate change, with implications for altered resources and thus plant competition. Few studies, however, have explored how  phenological traits relate to major traits associated with resource use and competition. Our work is the first to use data from global trait and phenology databases to test how woody plant budburst timing relates to other major functional traits and well established frameworks of species growth strategies. We found species that budburst earlier have traits associated with an acquisitive growth strategy, including shorter maximum heights, and high nitrogen leaves, while later budbursting species are taller with a low nitrogen content, indicative of a more conservative strategy that is favourable later in the growing season. In showing how budburst relates to traits from established functional trait frameworks, our results provide the foundation needed to predict how continued shifts in phenology may alter community dynamics under future conditions. % Ooh, nice ending!
\vspace{1.5ex}\\ 
We are happy to hear that our previous revisions addressed reviewers' concerns and that a third reviewer finds our work to be ``an interesting topic with current relevance" that ``advances the current understanding of budburst response to climate change."  In addressing reviewer comments, we have added additional context to our interpretation and discussion of the results. This includes adding details to the introduction and results elaborating on how we define small versus large cue responses and how it relates to the phenology literature. We also revised Figure 2 to clearly highlight our main results and include annotations to help guide readers in their interpretation. We extensively revised the structure of our writing throughout the manuscript, improving the flow of ideas and reader comprehension. Finally we added an additional subsection to the discussion and elaborated on the broader applications of our methods and general approach to other areas of ecology and conservation. %This seems good! Nice paragraph here. 
\vspace{1.5ex}\\
We believe the new manuscript has been greatly improved, as discussed in our point-by-point responses. The manuscript is 4536 words with a 301 word summary and three figures. This article is not under consideration for publication elsewhere. We hope you find it suitable for publication in \emph{Journal of Ecology}, and look forward to hearing from you. 
\vspace{1.5ex}\\
\noindent Sincerely, \\
\includegraphics[scale=.4]{letterhead/shot.png} \\ 
\noindent Deirdre Loughnan\\
\noindent Sentinels of Change Postdoctoral Fellow\\ %emw19Dec: Nice!
\noindent Hakai Institute $|$ Department of Zoology\\
\noindent University of British Columbia
\newpage

\end{document}

% do I have too much going on here? Chunky sentences? Can they be simplified?