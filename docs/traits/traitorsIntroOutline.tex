\documentclass{article}

% required 
\usepackage[hyphens]{url} % this wraps my URL versus letting it spill across the page, a bad habit LaTeX has

\usepackage{Sweave}
\usepackage{graphicx}
\usepackage{natbib}
\usepackage{amsmath}
\usepackage{textcomp}%amoung other things, it allows degrees C to be added
\usepackage{float}
\usepackage[utf8]{inputenc} % allow funny letters in citaions 
\usepackage[nottoc]{tocbibind} %should add Refences to the table of contents?
\usepackage{amsmath} % making nice equations 
\usepackage{listings} % add in stan code
\usepackage{xcolor}
\usepackage{capt-of}%alows me to set a caption for code in appendix 
\usepackage[export]{adjustbox} % adding a box around a map
\usepackage{lineno}
\linenumbers
% recommended! Uncomment the below line and change the path for your computer!
% \SweaveOpts{prefix.string=/Users/Lizzie/Documents/git/teaching/demoSweave/Fig.s/demoFig, eps=FALSE} 
%put your Fig.s in one place! Also, note that here 'Fig.s' is the folder and 'demoFig' is what each 
% Fig. produced will be titled plus its number or label (e.g., demoFig-nqpbetter.pdf')
% make your captioning look better
\usepackage[small]{caption}
\usepackage{xr-hyper} %refer to Fig.s in another document
\usepackage{hyperref}
\setlength{\captionmargin}{30pt}
\setlength{\abovecaptionskip}{0pt}
\setlength{\belowcaptionskip}{10pt}

% optional: muck with spacing
\topmargin -1.5cm        
\oddsidemargin 0.5cm   
\evensidemargin 0.5cm  % same as oddsidemargin but for left-hand pages
\textwidth 15.59cm
\textheight 21.94cm 
% \renewcommand{\baselinestretch}{1.5} % 1.5 lines between lines
\parindent 0pt		  % sets leading space for paragraphs
% optional: cute, fancy headers
\usepackage{fancyhdr}
\pagestyle{fancy}
\fancyhead[LO]{Draft 2022}
\fancyhead[RO]{Temporal Ecology Lab}
% more optionals! %

\graphicspath{ {./Figures/} }% tell latex where to find photos 
\externaldocument[supp-]{Synchrony_Manuscript_supp}

%%% end preambling. %%%

\begin{document}


\title{Traitors intro}
\date{\today}

\maketitle 

\begin{enumerate}

\item Forest communities contain diverse assemblages of species, the coexistence of which reflects both the ways in which species tolerate local environmental conditions and the outcome of various species interactions. 

\begin{enumerate}
\item As a result of these selective pressures and species specific responses, plant species within communities often have diverse growth strategies, varying in their resource use and competitive abilities. 
\item  A major access of diversity is in the timing of plant life history events, ranging from spring ephemeral species that leaf out early in the season to later leafing tree species that alter habitat conditions with increased canopy closure.
\item Despite considerable effort to study plant environmental cues and timing, why plants vary so much in the timing of their phenological events remains unclear.
\end{enumerate}

\item The timing of plant phenological events define the partitioning of resources across species over time, and therefore their temporal niche (Gotelli \& Granves 1996 - ch5).
\begin{enumerate}
\item Differences along this niche axis contribute to community assembly, defining the abiotic environment experienced during growth and biotic interactions -- competitive landscape and pressures from herbivory.
\item Studies of woody plants show diversity in species temporal niche e.g spring budburst and variability to specific environmental cues (Laube2014, Flynn2018)
\item While these studies do draw links to key environmental cues, the combined influence of abiotic and biotic cues, as well as relationships to species growth strategies and adaptive potential remains unclear.
\item Developing such a holistic framework of the drivers of species temporal niches differences is increasingly important, given the potential for climate change to shift species phenologies
%EMWJan8  -- I am not sure we toss in CC here (and very probably not cascading impacts), I think it may be too much for readers, so I commented out f the next two points ... you might be able to one of them in as a transition sentence before next part ... something more like 'understanding the drivers of species temporal niches is especially important now as CC is shifting phenology' or such. 
% \item Changes in climate are altering spp temporal niche, however, with earlier spring growth having cascading effects to ecological communities and spp interactions 
% \item But still have a limited mechanistic understanding of how specific spp will respond
\end{enumerate}

\item In many temperate forest communities, budburst occurs over the span of several weeks, reflecting differences in species morphology and physiology.
\begin{enumerate}
\item Species that bb in early in the spring = greater abiotic pressures, such as risk of false spring events and frost = affects early budbursting spp = potential loss of tissue, but benefit from more light availability
\item Late spp have greater selection from biotic pressures = less light available and competition for resources 
\item Differences in bb phenology within communities = important in shaping community dynamics including competition \& herbivory. 
\item But also the potential invasibility of a community, invasive spp tend to be early bb with the potential to fill vacant niche space early in the season. 
\end{enumerate}

\item Considerable work on the functional traits related to species growth strategies and competitive abilities — few studies include phenology
\begin{enumerate}
\item But the timing of species growth is likely to be related to species functional traits and their responses to abiotic and biotic pressures
\item Shrub and other woody understory species tend to budburst earlier in the growing season when light availability is greatest--- these species are shorter in stature, and may have leaf traits that are favourable to with standing late spring frost events, such as high leaf trichome densities, and more photosynthetic potential.
\item Canopy species often budburst later in the growing season---inherently taller, requiring greater wood densities, but greater access to light  throughout the growing season compensates for their slower growth rates.
\item But these later budbursting species must also be better competitors as they compete with more species than those that begin their growth early in the season.
\item Despite the intuitive nature of these trait associations, few studies to date have tested whether phenological traits follow similar gradients in growth strategies across diverse species
\end{enumerate}
 %EMWJan8 -- a lot of the points below don't connect to 'phenology (as day of year/early-late) x traits and how that connects to 2' -- you need a paragraph that covers this (things like spelling out that if early means before canopy closer than plants might be shorter etc. -- I know this is obvious to you, but will not be to your readers) before you dive into history of functional trait (LES etc.) -- I would expect those to be the next paragraph. So you have at least two paragraphs below to flesh out a little. 
\item Other commonly measured functional traits do follow gradients in their trait values, ranging from those associated with acquisitive (fast) growth to more conservative (slow) growth
\begin{enumerate}
\item Collectively, the responses of commonly measured leaf and morphological traits have lead to the development of the leaf economic spectrum. 
\item This framework represents and have been built into decades of research linking functional traits to species responses to abiotic and biotic factors and community assembly
\item But whether there are associations between other functional traits and the cues responses that define species temporal niche is unknown.
\end{enumerate}

\item Get into complexity of cues after hypotheses %EMWJan8  -- So, set up hypotheses HERE. They still seem down below ... And again, there's a lot of info below that needs to be fleshed into separate paragraphs ... 
% have Paragraph on cues as creating early to late --> explain cues -- pt 6? Keep a -c as 5
\begin{enumerate}
% add definitions for the cues
\item Previous studies have shown 3 cues are most important for spp responses:
\begin{enumerate}
\item Chilling - the period of cold temperatures from late fall to late winter, releases buds from dormancy
\item Forcing - the occurrence of warm temperatures in spring that initiate bud development 
\item Photoperiod - daylength
\end{enumerate}
%EMWJan8 -- below stuff is good, but I think it belongs elsewhere in intro? 
\item field obs of phenology are highly variable — but under controlled environments and set cues, bb is highly predictable
\item traits themselves can be highly variable, both across and within spp - Violle paper ‘viva la variability’
\item trait ecology’s goal = predict sp-level characteristics by traits alone — how well we can do this to highly variable and species specific traits like phenology is unclear
\item Potential to use phenological data from controlled environment studies to identifying the relationship between sp cue responses and traits
\end{enumerate}

\item Set up hypotheses...
% I was concerned that it would not make sense to introduce the hypotheses as they were written before with the cues before defining the cues in the above paragraph, so I moved the hypotheses below
\begin{enumerate}
\item We predict that spp with traits associated with acquisitive growth (high SLA, high LNC, short heights, small seeds) will have will have cue requirements associated with low forcing, chilling, and photoperiod cue requirements
\item Spp that are better competitors with conservative growth and later budburst, with low SLA and LNC, tall heights, and large seeds, will have phenological response associated with high forcing, chilling, and photoperiod requirements
\end{enumerate}

\item Here's what we do here. %EMWJan8 -- there's repetition here, including model stuff that belongs in model section that comes next. 
\begin{enumerate}
\item we test for associations between plant phenological responses to environmental cues and common functional traits 
\item use available trait data from trait databases with bb data from controlled environment studies of woody plant species from the OSPREE database.
\item We focus on the effects of forcing, chilling, and photoperiod cues and four easy to measure traits — SLA, LNC, height, \& seed mass
\item Our model attributes phenological variation (day of bb) to species’ trait values while including residual variation from species (partial-pooling).
\item When traits explain a significant portion of the variation, spp will explain only a small amount — may be able to predict spp growth strategies and phenological responses from trait values.
\item Potential to use phenological data from controlled environment studies to identifying the relationship between sp cue responses and traits
\end{enumerate}

\item Our model ...
\begin{enumerate}
\item Our model attributes phenological variation (day of bb) to species’ trait values while including residual variation from species (partial-pooling).
\item When traits explain a significant portion of the variation, spp will explain only a small amount — may be able to predict spp growth strategies and phenological responses from trait values.
\end{enumerate}
\end{enumerate}

Need to fit in into intro, not sure where:
\begin{enumerate}
\item Cues address phenological variability
\item Be sure to clearly set up acquisitive vs. conservative
\end{enumerate}

Stuff we had, but could cut:
\begin{enumerate}
\item details of phenological responses - ectodormancy transition to endodormancy -- Cutting this, too much other content
\item detailed definition of forcing, chilling, photoperiod
\end{enumerate}

Fig 1:
i) Can you confirm the slopes (when trait effect = 0) are constant across the top conceptual panels? If not, we should make them that way I think to minimize what changes across them?
DLDec15: Yes they are, the betaChill slope is always -2, betaForceSp =5 and betaPhotoSp = 1 for each plot
%EMWJan8 -- I am pretty sure that I replied to this somewhere before; in my previous notes? (Basically we should keep the cue constant across the three panels so the slopes are also constant ...) But if I just replied in a really detailed way in my head, let me know and I can add to https://github.com/lizzieinvancouver/ospree/issues/444

ii) Make sure the figure in the Supp that is similar has the same y axis scale -
DLDec15: I will fix that in the next draft for sure

\end{document}

