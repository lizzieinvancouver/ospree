\documentclass[11pt,letterpaper]{article}
\usepackage[top=1.00in, bottom=1.0in, left=1in, right=1in]{geometry}
\usepackage{textcomp}
\usepackage{amsfonts}
\usepackage{verbatim}
\usepackage[english]{babel}
\usepackage{pifont}
\usepackage{color}
\usepackage{setspace}
\usepackage{lscape}\parskip=5pt
\usepackage{gensymb} % You have to have this to use \degree
\usepackage{float}
\usepackage{latexsym}
\usepackage{url}

% Reference Supp labels
\usepackage{zref-xr}
\zxrsetup{toltxlabel=true, tozreflabel=false}
\zexternaldocument*{winefuture_supp}
\usepackage{epsfig}
\usepackage{graphicx}
\usepackage{amssymb}
\usepackage{amsmath}
\usepackage{soul} % Need for st command (sout for strikethrough does not work with math lib loaded)

\usepackage{caption}
\usepackage{lineno}
\usepackage[utf8]{inputenc}
\usepackage{sectsty,setspace,natbib}
\usepackage{graphicx}
\usepackage{latexsym,epsf,rotating}
\usepackage{epstopdf}

\linespread{1.2} % was 1.66 for double-spaced 
% \raggedright
\setlength{\parindent}{0.5in}

\setcounter{secnumdepth}{0}

\pagestyle{empty}

\renewcommand{\tableofcontents}{}

\parskip=5pt
\pagenumbering{arabic}
\pagestyle{plain}

\usepackage{fancyhdr}
\pagestyle{fancy}
\fancyhead[LO]{OSPREE BB data}
\fancyhead[RO]{\today}
% put in your own RH (running head)

\def\labelitemi{--}
\setlength\parindent{0pt} % make document noindent all the way through

\begin{document}
\begin{flushright}
Version dated: \today
\end{flushright}
%\bigskip
%\noindent RH: All cues drive temperate tree phenology 
\thispagestyle{empty}
\bigskip
\medskip
\begin{center}
\noindent{\Large {\bf OSPREE \\ Notes on what we want to do before moving on...}}\\
\vspace{2ex}
\bigskip
\end{center}

%\linenumbers
% Check what spp. in PEP725
% Re-do outline
% Check for correlations and trade-offs across species
This file started with the following setup: given the OSPREE data what would you want to know? Or, phrased differently, what do you expect others would immediately do with these data and of that list which items would we have wished we had done? 

We (Ailene, Cat, Dan, Nacho, Lizzie) discussed this in August 2018 and came up a following list of projects/tasks. Most of these tasks could fit into one or more paper (see further below; note how one project may show up in multiple paper conceptions), but step 1 seems to be to decide which ones we want to do and get them started. So here's the overall list of what we'd need to do to tackle \emph{everything} that we came up with.\footnote{Note that we probably don't want or need to do everything.}


\subsection{Stuff we should do ...} 
Below updated at July 2019 retreat:
\begin{enumerate}
\item Bigger tasks (to divvy up)
\begin{enumerate}
\item \% budburst model (Ailene expressed interest; Dan is working on similar modeling approaches)
\begin{enumerate}
\item Step 1: Find out if we have fake data (if not, make fake data)
\item Side note: Data should be pretty clean, \% BB is not thrown out until step 6 in BB cleaning.
\end{enumerate}
\item Range questions: Everyone!
\begin{enumerate}
\item Calculate exact geographical position within (or beyond) range of each species in each study
\item Calculate climatic position within (or beyond) range of each species in each study 
\item Provenance questions: Do cues vary across provenance? Ideally in some way (northern or coastal or up a mountain etc.) but barring that do they vary? Or even how much does provenance explain days to budburst or such?
\begin{enumerate}
\item Elevation models? (Cat)
\item Effects of coast models? (Cat)
\item Get trait data (and correlate cues with trait data)\footnote{For here and for all our questions, we'll want to think carefully about not running so MANY models that we are sure to find something, AKA sure to find something possibly spurious.} Darwin, Deirdre working on this.
\end{enumerate}
\begin{enumerate}
\item wood anatomy
\item leaf vein anatomy (Sack papers?)
\item Leaf Economic Spectrum: SLA and LDMC and C:N
\item trichome density: higher density correlates with earlier leafout cues
\item hysteranthy
\item seed dormany characteristics
\item shade tolerance
\item height
\item cold/thermal tolerance 
\item growth rate and longevity, could use pioneer ... see Laube \emph{et al.} 2012 (faster growth rate, should have cues to vs. earlier active)
\item discussion item: root traits, probably important, hard to get
\item Traits we don't think we want: 
\item Traits we're not sure whether we want or not: seed mass
\end{enumerate}
\begin{enumerate}
\item Step 1: Identify traits of interest
\item Step 2: Select species
\item Step 3: Find traits
\item Step 4: Settle on hypotheses to test once we know the dat
\end{enumerate}
\item Calculate forcing and chilling sensitivities from PEP725 data for OSPREE species (open option)
\item Calculate delays in advances in OSPREE species from PEP725 data (Ben is working on something simiilar)
\item Get phylogeny for our species, add it to basic BB model (Nacho, in BRMS, also do phylogenetic signal)
\end{enumerate}
\end{enumerate}
\item Do we want to analyze the performance data? 
\end{enumerate} 

\subsection{Stuff we said we wanted to do and did!}
Updated July 2019
\begin{enumerate}

\item Tasks related to BB model (probably do not need to divvy up right now, as we will get these done methinks)
\begin{enumerate}
\item BB models with Weinberger
\item Revisit non-linear BB models: No sign of non-linearities in our data much, at least for chilling. 
\item BB models: Does temperature treatment covary with photoperiod? ... see issues \#220 and \#235 at least. Also, countinxns code does this now!
\item BB model interpretation: Apply 1 C of warming (across year, in certain seasons) ... so very done!
\end{enumerate}
\item Other stuff ... ?
\end{enumerate}

\subsection{Lizzie's notes from way back when ...}

{\bf Directions I can think of that we could take these budburst data:}
\begin{enumerate}
\item \% budburst models (Question: Do we need non-linear $\beta$ models for this? Or does link in $\beta$ models cover this nonlinearity?)
\item Examination of what can predict variation in cues (pro: lots of people will ask about this so we'll have an answer; con: need to set it up so it is not a fishing expedition):
\begin{enumerate}
\item Range size
\item Climatic niche and phenological response? Any predictions? (Size of nice relates to variance in responses; bigger niche more flexible response)
\item Range related questions: I think range size and cue variation is interesting, relatedly (perhaps I am influence by having just looked at Picea abies), it would be interesting to see if there are differences between studies within normal range provenances and outside of them.
\item Latitude and provenance questions (e.g., our current latitude model, also, for studies with just provenance we could fit $BB~provenance+error$)
\item Some climatic attribute of range
\item How far outside their range the cutting was taken + some climatic attribute of range
\item Traits (would need to collect these data ourselves and then see if we trust them)
\item Other cues (do cues trade-off or correlate across species)?
\item Phylogeny
\item None of the above, it's about study design or such ... see below (below text also moved to budburst.tex) % note the model with study ID did not change estimates much
\end{enumerate}
\item Understanding our results (cue estimates) in relation as to when species may delay or do weird things (pro: it's an obvious and important question, and one we have a lot of expertise to answer; cons: we know that real phenological models are more complex, so what can we really offer?) 
\begin{enumerate}
\item How similar are our estimates from sensitivities you could estimate from long-term data (I think we can do this for chilling and forcing using PEP725 data).
\item Use species from \citep{fu2015} to illustrate what 1$\degree$C would mean if:
\begin{enumerate}
\item Evenly applied across year
\item Applied to only certain seasons ... 
\end{enumerate}
\item We could see if we can predict the delay seen in PEP725 data? 
\end{enumerate}
\item Look into two major areas that could cause variation we see given the data we have ... (pros: we have the data to do this; cons: may be too similar to limiting cues MS). There are two (three) big sources of reasons for the variation:
\begin{enumerate}
\item Methodology
\begin{enumerate}
\item Imputation model
\item Weinberger method! 
\item Model with experimental versus field chilling... (do similar to Weinberger? We could code each study as one of three types of chill: all field chill, all experimental chill, field+exp, or what about fitting a model with studies with no field chilling? Or look at studies that calculated their own units?)
\item Non-linear model?
\item How to do photoperiod (i.e., does temperature co-vary with photoperiod)?
\end{enumerate}
\item Species, population-level etc.
\begin{enumerate}
\item Latitude model
\item Provenance model
\item Trade-offs and correlations in cues among species 
\item Coastal versus non-coastal
\item Continent of origin?
\end{enumerate}
\item Also, sort of climate (i.e., the same tree planted in Finland and southern low-elevation Germany will leaf out first at the German site) ... but we sort of deal with that via the study design, no?
\end{enumerate}
\end{enumerate}



%=======================================================================
% References
%=======================================================================
\newpage
\bibliography{..//..//refs/ospreebibplus.bib}
\bibliographystyle{apa}


\end{document}



%=======================================================================
% to-do listing
%=======================================================================



